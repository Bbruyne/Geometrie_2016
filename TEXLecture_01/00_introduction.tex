\documentclass{beamer}

\usepackage[utf8]{inputenc}
\usepackage{hyperref}
\usepackage{pgfpages}
\usepackage{dirtree}
\setbeameroption{show notes on second screen =left} %hide nodes, show only notes, show notes on second screen = left.
\setbeamertemplate{note page}[plain]
\AtEndNote{\vfill \begin{center} mm:hh \end{center}}
\newcommand{\notedir}[1] {
  \note{\dirtree{#1}}}
  
\definecolor{links}{HTML}{FF8000}
\hypersetup{colorlinks,linkcolor=,urlcolor=links}

\def \nom {Benjamin De Bruyne} %nom de l'étudiant-moniteur
\def \email {benjamin.debruyne@student.ulg.ac.be} %email de l'étudiant-moniteur

\title{Géométrie.}
    \author{\nom}
    \institute
    {
      Faculté des Sciences Appliquées.\\
      Université de Liège.
    }
    
    \date{Août 2016}
    
\begin{document}  
    \beamertemplatenavigationsymbolsempty
    
    \frame{\titlepage
	    \notedir{%
	      .1 Introduction.
	      .2 Mot d'accueil.
	      .2 Se présenter.
	      .3 Etudiant en ingénieur~civil..
	      .3 Réussi examen entrée..
	      .3 Viens pour donner trucs et astuces..
	      .2 Présenter module Géométrie.
	      .3 Aujourd'hui.
	      .4 Questions des examens récents..
	      .4 Questions sur géométrie synthétique dans l'espace..
	      .3 Mercredi.
	      .4 Questions sur géométrie synthétique plane..
	      .4 Questions sur géométrie vectorielle..
	      .3 Jeudi.
	      .4 Questions sur géométrie analytique dans l'espace..
	      .4 Questions sur géométrie analytique plane..	      
	    }
	  }
    
    \frame{
	   \frametitle{Liens utiles. (Cliquez sur le texte)}
	   \begin{enumerate}
	      \item{\href{http://www.facsa.ulg.ac.be/upload/docs/application/pdf/2012-07/geometrie_-_synthese.pdf}{Théorie}},
	      \item{\href{http://www.montefiore.ulg.ac.be/~boigelot/cours/adm/}{Examens d'anciennes sessions}},
	      \item{Questions : \href{mailto:\email}{\email}}.
	   \end{enumerate} 	   
	  }
	  
\end{document}
