\documentclass[10pt]{beamer}

\usepackage[utf8]{inputenc}
\usepackage{tcolorbox}
\usepackage{tikz}
\usepackage{tikz-3dplot}
\usetikzlibrary{intersections,calc,,angles,quotes,through}
\usepackage{amsmath}
\usepackage{graphicx}
\usepackage{cases}
\def \heart {\textcolor{blue}{$\heartsuit$} }
\def \C {\mathcal{C}}
\def \orthog {\underline{\perp}}
\def\arcos{\operatorname{arcos}}
\def \deg {^{\circ}}

\newcommand{\vect}[1] {
  \overrightarrow{#1}}

\tcbset{%
	basic/.style={colframe=black,
		      colback=white,
		      top= 0mm,
		      bottom = 2mm,
		      boxsep=0mm
		      }
}
\tikzset{
    invisible/.style={opacity=0},
    visible on/.style={alt={#1{}{invisible}}},
    alt/.code args={<#1>#2#3}{%
      \alt<#1>{\pgfkeysalso{#2}}{\pgfkeysalso{#3}} % \pgfkeysalso doesn't change the path
    },
  }

    
\begin{document}  
    \beamertemplatenavigationsymbolsempty
    \setlength{\abovedisplayskip}{0pt}
    \setlength{\belowdisplayskip}{0pt}
    \frame{
	  
	  \frametitle{Q5 Septembre 2001.}
	  \renewcommand{\theenumi}{\alph{enumi})}
	  \begin{enumerate}
	   \item Quelles sont les coordonnées du pied $Q$ de la perpendiculaire abaissée du point
		 $P (\alpha, \alpha + 1, \alpha - 1)$ sur le plan $\pi$ d’équation $2x + \alpha y + \alpha z + \alpha^3 + 4 = 0$ ?
	   \item Montrer que ces pieds, lorsque $\alpha$ parcourt $\mathbb{R}$, sont tous situés sur une même
		 droite $d$, dont on déterminera des équations.
	  \end{enumerate}

	  \vfill
	  
	  \pause
	  
	   \begin{tcolorbox}[basic] \smallskip
	     \centering\underline{Traduction}
	     \begin{columns}[c]
	     \column{.5\textwidth}\centering 
	     \renewcommand{\theenumi}{\alph{enumi})}
	     \begin{enumerate}
	      \item $PQ \equiv \begin{cases}
	                        \vect{n_\pi}, \\
	                        \ni P.
	                       \end{cases}$ \\ \medskip
	            $\phantom{P}Q = PQ \cap \pi$.


	     \end{enumerate}
	     \column{.5\textwidth}\centering 
	     \vspace{-7mm}
	     \begin{enumerate}
	      \item[b).] Eliminer le paramètre $\alpha$.

	     \end{enumerate}

	  \end{columns}
	  \end{tcolorbox}
    }

    \frame{ 
	  % résolution ex1

		
				  \begin{tcolorbox}[basic] 
				      
				    \smallskip
				    \centering\underline{Traduction}
				    \begin{columns}[c]
				    \column{.5\textwidth}\centering 
				    \renewcommand{\theenumi}{\alph{enumi})}
				    \begin{enumerate}
				      \item $PQ \equiv \begin{cases}
							\vect{n_\pi}, \\
							\ni P.
						      \end{cases}$ \\ \medskip
					    $\phantom{P}Q = PQ \cap \pi$.


				    \end{enumerate}
				    \column{.5\textwidth}\centering 
				    \vspace{-7mm}
				    \begin{enumerate}
				      \item[b).] Eliminer le paramètre $\alpha$.

				    \end{enumerate}

				    \end{columns}
				    \end{tcolorbox}
		
		
		
		\centering\underline{Résolution}\\ \medskip
		
		$\pi \equiv 2x + \alpha y + \alpha z + \alpha^3 + 4 = 0 \qquad \rightarrow \qquad \vect{n_\pi}=(2,\alpha,\alpha)$. \bigskip\flushleft
		\begin{columns}
		
		\column{.5\textwidth}\centering 
		\begin{align*}
		 PQ &\ \equiv \begin{cases}
		            \vect{n_\pi}, \\
			    \ni P(\alpha, \alpha + 1, \alpha - 1).
		           \end{cases} \\
		    &\ \equiv \frac{\alpha(x-\alpha)}{2}=y-\alpha-1=z-\alpha+1.
		\end{align*}\bigskip
		
		\column{.5\textwidth}\centering
		\begin{figure}[h]
				  \begin{tikzpicture}[scale=0.85]
			          %projection ($(X)!(B')!(B)$)
			          %nommer chemin 'name path
			          %intersection \path [name intersections={of=d and gb,by=G}];
			          
			          %\draw[help lines] (-3,-3) grid (3,3);
			          \draw[dotted] (2,0,0) -- (2,0,2) node[above left]{$\pi$} -- (-2,0,2) -- (-2,0,0);
			          \coordinate[label=above:$P$](P) at (0,0.8,1);
			          \fill (P) circle (0.025);
			          \draw[->] (-1.5,0,1) -- (-1.5,0.5,1) node[left]{$\vect{n_\pi}$};
			          \draw (P) -- (0,0,1) node[left]{$Q$};
			          \draw (0,0,1) -- (0,-0.5,1);
			          \draw (0,0,0.85) -- (0,0,1.15) (-0.08,0,1) -- (0.08,0,1);
			          
				  \end{tikzpicture}
				  \end{figure}
		\end{columns}
		
		$Q =  PQ\ \cap \ \pi = \begin{cases}
		                     \frac{\alpha(x-\alpha)}{2}=y-\alpha-1, \\
		                     y-\alpha-1=z-\alpha+1, \\
		                     2x + \alpha y + \alpha z + \alpha^3 + 4 = 0.
		                    \end{cases}
		                    \hspace{-2mm} \rightarrow  Q(-2,\frac{2-\alpha^2}{2},-\frac{2+\alpha^2}{2})$.

		\hfill $\qed(a)$
		%\centering\noindent\rule{2cm}{0.4pt}
	
    }
	  
	  
    \frame{ 
	  % résolution ex1

		
				  \begin{tcolorbox}[basic] 
				      
				    \smallskip
				    \centering\underline{Traduction}
				    \begin{columns}[c]
				    \column{.5\textwidth}\centering 
				    \renewcommand{\theenumi}{\alph{enumi})}
				    \begin{enumerate}
				      \item $PQ \equiv \begin{cases}
							\vect{n_\pi}, \\
							\ni P.
						      \end{cases}$ \\ \medskip
					    $\phantom{P}Q = PQ \cap \pi$.


				    \end{enumerate}
				    \column{.5\textwidth}\centering 
				    \vspace{-7mm}
				    \begin{enumerate}
				      \item[b).] Eliminer le paramètre $\alpha$.

				    \end{enumerate}

				    \end{columns}
				    \end{tcolorbox}
		
		
		
		\centering\underline{Résolution}\\ \medskip
		$Q(-2,\frac{2-\alpha^2}{2},-\frac{2+\alpha^2}{2})\qquad \rightarrow \qquad \begin{cases}
											    x = -2, \\
											    y = \frac{2-\alpha^2}{2}, \; \text{ ($\dagger$)}\\
											    z = -\frac{2+\alpha^2}{2}. 
											    \end{cases}$ \bigskip
											    
		$(\dagger)\quad \alpha^2 = -2y + 2  \qquad \rightarrow \qquad d \equiv      \begin{cases}
											    x = -2, \\
											    y = z + 2. 
											    \end{cases}$  \\ \bigskip
											    
		$d$ est une droite car intersection de 2 plans. \\
											    \hfill $\qed(b)$
		
		%\centering\noindent\rule{2cm}{0.4pt}
	
    }
  
\end{document}
