\documentclass[10pt]{beamer}

\usepackage[utf8]{inputenc}
\usepackage{pgfpages}
\usepackage{dirtree}
\setbeamertemplate{note page}[plain]
\setbeameroption{show notes on second screen =left}
\AtEndNote{\vfill \begin{center} mm:hh \end{center}}
\newcommand{\notedir}[1] {
  \note{\dirtree{#1}}}
\def \ion {$^{\circ}$ }
\usepackage{tcolorbox}
\usepackage{tikz}
\usepackage{tikz-3dplot}
\usetikzlibrary{intersections,calc,,angles,quotes,through}
\usepackage{amsmath}
\usepackage{graphicx}
\usepackage{cases}
\def \heart {\textcolor{blue}{$\heartsuit$} }
\def \C {\mathcal{C}}
\def \orthog {\underline{\perp}}
\def\arcos{\operatorname{arcos}}
\def \deg {^{\circ}}

\newcommand{\g}[1] {
  \mathbf{#1}}
  
\newcommand{\vect}[1] {
  \overrightarrow{#1}}

\tcbset{%
	basic/.style={colframe=black,
		      colback=white,
		      top= 0mm,
		      bottom = 2mm,
		      boxsep=0mm
		      }
}
\tikzset{
    invisible/.style={opacity=0},
    visible on/.style={alt={#1{}{invisible}}},
    alt/.code args={<#1>#2#3}{%
      \alt<#1>{\pgfkeysalso{#2}}{\pgfkeysalso{#3}} % \pgfkeysalso doesn't change the path
    },
  }

  \def\enonce{ \frametitle{Q3 Septembre 2001. }
	  
	  On donne quatre points $A,B,C,D$ dans le plan. \\
	  \begin{enumerate}
            \renewcommand{\theenumi}{\alph{enumi})}
	   \item Montrer que si $M$ est un point du plan, le vecteur \\
	   $$\g{v}=4\vect{MA}+3\vect{MB}-5\vect{MC}-2\vect{MD}$$ \\
	   est indépendant du point $M$.
	   \item Montrer que si le vecteur $\g{v}$ de $a).$ est nul, le nombre \\
	   $$x=4|\vect{MA}|^2 + 3|\vect{MB}|^2 - 5|\vect{MC}|^2 -2 |\vect{MD}|^2$$ \\
	   est aussi indépendant du point $M$.
         \end{enumerate}}

       
        \def\hypotheses{ \underline{Hypothèses}
		      \begin{enumerate}
		      \item $\g{v}=4\vect{MA}+3\vect{MB}$ \\ 
					  \smallskip$\phantom{{}=1} -5\vect{MC}-2\vect{MD}$.
		      \item $\g{v}=\g{0}$, \\ \smallskip
		      
			    $x=4|\vect{MA}|^2 + 3|\vect{MB}|^2$\\ \smallskip$\phantom{{}=1} - 5|\vect{MC}|^2 -2 |\vect{MD}|^2$.
		      \end{enumerate}
                    }

                    \def\these{\underline{Thèse} 
		      \begin{enumerate}
                      \renewcommand{\theenumi}{\alph{enumi})}
                      \item $\g{v}$ indépendant de $M$,
		       \item $x$ indépendant de $M$.
		      \end{enumerate}}
       
    
\begin{document}  
    \beamertemplatenavigationsymbolsempty
    \setlength{\abovedisplayskip}{0pt}
    \setlength{\belowdisplayskip}{0pt}
    \frame{
	  
	  \enonce

	  \vfill
	  
	  \pause
	  % hypothèses et thèse
	  \begin{tcolorbox}[basic] 
	      \begin{columns}[t]
		 
		 \column{.55\textwidth}\centering
		      
		      \hypotheses
		  
		  \column{.5\textwidth}\centering
		      
		      \these

	      \end{columns}
	  \end{tcolorbox}
	  \notedir{%
	.1 Énoncé.
	.2 Hypothèses..
	.2 Thèse..
	.2 Dessin non nécessaire car aucune figure géométrique..
	}
    }

    \frame{ 
	  % résolution ex1
		\begin{tcolorbox}[basic] 
	      \begin{columns}[t]
		 
		 \column{.55\textwidth}\centering

                 \hypotheses
		  
		  \column{.5\textwidth}\centering
		      
		   \these
		
	      \end{columns}
	  \end{tcolorbox}
		
		\centering\underline{Résolution}\\ \flushleft
		
		\begin{enumerate}
		 \item \begin{align*}
			\g{v}=&\ 4\vect{MA} + 3\vect{MB} - 5\vect{MC} - 2\vect{MD},\\
			     =&\ 4\vect{MA} + 3(\vect{MA} + \vect{AB}) -5(\vect{MA} - \vect{AC}) - 2(\vect{MA} -\vect{AD}), \\
			     =&\ 3\vect{AB}  - 5\vect{AC}  -2\vect{AD}.
		       \end{align*} \hfill $\qed(a)$
		\end{enumerate}	
    \notedir{%
	   .1 Prouver thèse.
	   .2 $\g{v}$ indépendant de $M$.
	   .3 Élément de théorie.
	   .4 Décomposition de vecteurs..
	   .3 Résolution..
	   .4 Décomposer les vecteurs pour n'en avoir qu'un qui\\ \hspace{5mm} contient $M$..
	   .5 Ce dernier s'annule car somme\\ \hspace{5mm} coefficient vaut $0$..
	   .6 $\g{v}$ indépendant de $M$..
	   }
    }
    
    \frame{ 
	  % résolution ex1
		\begin{tcolorbox}[basic] 
	      \begin{columns}[t]
		 
		 \column{.55\textwidth}\centering
		      
		      \hypotheses
		  
		  \column{.5\textwidth}\centering
		      
		      \these
		
	      \end{columns}
	  \end{tcolorbox}
		
		\centering\underline{Résolution}\\ \flushleft
		
		\begin{enumerate}
		 \item[2.] \begin{align*}
                        x=\ &4|\vect{MA}|^2 + 3|\vect{MB}|^2 - 5|\vect{MC}|^2 -2 |\vect{MD}|^2, \\[0.3em]
                         =\ &4\vect{MA}\cdot\vect{MA} + 3\vect{MB}\cdot\vect{MB} - 5\vect{MC}\cdot\vect{MC} -2\vect{MD}\cdot\vect{MD}, \\[0.3em]
                         =\ &4\vect{MA}\cdot\vect{MA} + 3(\vect{MA} + \vect{AB})\cdot(\vect{MA} + \vect{AB})\\
                            &- 5(\vect{MA} + \vect{AC})\cdot(\vect{MA} + \vect{AC})\\
                            & -2(\vect{MA} + \vect{AD})\cdot(\vect{MA} + \vect{AD}), 
		       \end{align*}
		\end{enumerate}	
    \notedir{%
	   .1 Prouver thèse.
	   .2 $\g{v}=0 \rightarrow x$ indépendant de $M$. 
	   .3 Élément de théorie.
	   .4 Décomposition de vecteurs..
	   .3 Résolution..
	   .4 Décomposer les vecteurs pour n'en avoir qu'un qui\\ \hspace{5mm} contient $M$..
	   .5 Le carré d'un vecteur est le produit scalaire de\\ \hspace{5mm} celui-ci par lui-même..
	   }
    }

        \frame{ 
	  % résolution ex1
		\begin{tcolorbox}[basic] 
	      \begin{columns}[t]
		 
		 \column{.55\textwidth}\centering
		      
		  \hypotheses
		  
		  \column{.5\textwidth}\centering
		      
		  \these
		
	      \end{columns}
	  \end{tcolorbox}
		
		\centering\underline{Résolution}\\ \flushleft
		
		\begin{enumerate}
		 \item[2.] \begin{align*}
                        x=\ &4\vect{MA}\cdot\vect{MA} + 3(\vect{MA}\cdot\vect{MA} + 2\vect{MA}\cdot\vect{AB} + \vect{AB}\cdot\vect{AB}) \\
                            & - 5(\vect{MA}\cdot\vect{MA} +2\vect{MA}\cdot\vect{AC} + \vect{AC}\cdot\vect{AC})\\
                            & -2(\vect{MA}\cdot\vect{MA} + 2\vect{MA}\cdot\vect{AD} + \vect{AD}\cdot\vect{AD}), \\[0.3em]
			 =\ &2\vect{MA}\cdot(3\vect{AB}  - 5\vect{AC}  -2\vect{AD}) + 3\vect{AB}\cdot\vect{AB} - 5\vect{AC}\cdot\vect{AC} -2\vect{AD}\cdot\vect{AD}, \\[0.3em]
			 =\ &2\vect{MA}\cdot\vect{\g{v}} + 3\vect{AB}\cdot\vect{AB} - 5\vect{AC}\cdot\vect{AC} -2\vect{AD}\cdot\vect{AD},\\[0.3em]
			 =\ &3\vect{AB}\cdot\vect{AB} - 5\vect{AC}\cdot\vect{AC} -2\vect{AD}\cdot\vect{AD}. 
		       \end{align*} \hfill $\qed(b)$
		\end{enumerate}	
    \notedir{%
	   .1 Prouver thèse.
	   .2 $\g{v}=0 \rightarrow x$ indépendant de $M$. 
	   .3 Élément de théorie.
	   .4 Décomposition de vecteurs..
	   .3 Résolution..
	   .4 Décomposer les vecteurs pour n'en avoir qu'un qui\\ \hspace{5mm} contient $M$..
	   .5 Le carré d'un vecteur est le produit scalaire de\\ \hspace{5mm} celui-ci par lui-même..
           .5 Le produit scalaire est distributif comme le \\ \hspace{5mm} produit entre deux nombres.
	   .6 Comme $\g{v}=0$, $x$ indépendant de $M$..
	   }
         }

             \frame{ 
	  \enonce

	  \vfill\center
	  \underline{Résumé}\\ \flushleft
          \begin{itemize}
	  \item Montrer qu'une expression vectorielle est indépendante d'un point = décomposer vecteurs pour n'en avoir plus qu'un qui contient ce point.
	  \item $|\vect{AB}|^2=\vect{AB}\cdot\vect{AB}$.
          \end{itemize}
    }
  
\end{document}

%%% Local Variables:
%%% mode: latex
%%% TeX-master: t
%%% End:
