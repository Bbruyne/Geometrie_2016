\documentclass[10pt]{beamer}

\usepackage[utf8]{inputenc}
\usepackage{pgfpages}
\usepackage[french]{babel}
\usepackage{dirtree}
\setbeamertemplate{note page}[plain]
\AtEndNote{\vfill \begin{center} mm:hh \end{center}}
\newcommand{\notedir}[1] {
  \note{\dirtree{#1}}}
\def \ion {$^{\circ}$ }
\usepackage{tcolorbox}
\usepackage{tikz}
\usepackage{tikz-3dplot}
\usetikzlibrary{intersections,calc,,angles,quotes,through}
\usepackage{amsmath}
\usepackage{graphicx}
\usepackage{cases}
\def \heart {\textcolor{blue}{$\heartsuit$} }
\def \C {\mathcal{C}}
\def \orthog {\underline{\perp}}
\def\arcos{\operatorname{arcos}}
\def \deg {^{\circ}}

\newcommand{\vect}[1] {
  \overrightarrow{#1}}

\tcbset{%
	basic/.style={colframe=black,
		      colback=white,
		      top= 0mm,
		      bottom = 2mm,
		      boxsep=0mm
		      }
}
\tikzset{
    invisible/.style={opacity=0},
    visible on/.style={alt={#1{}{invisible}}},
    alt/.code args={<#1>#2#3}{%
      \alt<#1>{\pgfkeysalso{#2}}{\pgfkeysalso{#3}} % \pgfkeysalso doesn't change the path
    },
  }
  \def\enonce{ \frametitle{Exercice sur les points particuliers du triangle. }
	  %\renewcommand{\theenumi}{\alph{enumi})}
	  Dans un triangle $ABC$ avec $O$ le centre du cercle circonscrit au triangle, $G$ le
	  centre de gravité, montrer que si \\
	  $$H = O + 3\vect{OG},$$  \\
	  alors $H$ est l'orthocentre du triangle.
    }

    \def\hypotheses{ \underline{Hypothèses} 
		      \begin{enumerate}
		      \item $H = O + 3\vect{OG}$,
		      \item $G$ centre de gravité.
		      \end{enumerate}
                    }
                    \def\these{\underline{Thèse} \\
		      \smallskip
		      $H$ orthocentre.
                      }
\begin{document}  
    \beamertemplatenavigationsymbolsempty
    \setlength{\abovedisplayskip}{0pt}
    \setlength{\belowdisplayskip}{0pt}
    \frame{
	  \enonce
	 
	  \vfill
	  
	  \pause
	  % hypothèses et thèse
	  \begin{tcolorbox}[basic] 
	      \begin{columns}[t]
		 
		 \column{.5\textwidth}\centering
		     
\hypotheses
		  
\column{.5\textwidth}\centering

		      \these
		
	      \end{columns}
	  \end{tcolorbox}
	  \notedir{%
	.1 Énoncé.
	.2 Hypothèses (non visibles sur le dessin)..
	.2 Thèse..
	.2 Grand dessin.. 
	}
    }

    \frame{ 
	  % résolution ex1
	  \begin{columns}[t]
		\column{.46\textwidth}\centering 
		

			\underline{Dessin}\\
				  \vspace{-15mm}
				  \begin{figure}[h]
				  \begin{tikzpicture}[scale=0.8]
			          %projection ($(X)!(B')!(B)$)
			          %nommer chemin 'name path
			          %intersection \path [name intersections={of=d and gb,name=G}];
			          %animation  \draw[visible on=<1>] 
				  %           \draw[visible on=<{2,4}>]
				  %angle arc[radius = 6mm, start angle= 180, end angle= 225] node [below left,pos=0.3]{$\alpha$}
				  %angle \pic [draw,"$\alpha$", angle eccentricity=1.5] {angle = A'--A--B};
				  %perpendiculaire ($(A')!3cm!-90:(A)$)
				  %cercle par point \node [draw] at (A) [circle through=(B)] {};
				  
				  %CERCLE et triangle
				  \coordinate[label=below:$O$] (O) at (0,0);
				  \coordinate[label=above right:$B$] (B) at (70:2);
				  \coordinate[label=above right:$C$] (C) at (10:2);
				  \coordinate[label=above left:$A$] (A) at (170:2);
				  \draw[name path =cercle] (O) circle (2);
				  \draw (A) -- (B)--(C) --cycle;
				  \fill (O) circle (0.025);
				  
				  %G
				  \coordinate[label=above right:$G$] (G) at ($.333*(A)+.333*(B)+.333*(C)$);
				  \fill (G) circle (0.025);
				  %H
				  \path[name path=BH] (B) -- +($2*(B)-2*(A)!(B)!(C)$);
				  \path[name path=CH] (C) -- +($2*(A)!(C)!(B)-2*(C)$);
				  \path [name intersections={of=BH and CH,by=H}];
				  \coordinate[label=above right:$H$] () at (H);
				  \fill(H) circle (0.025);
				  %M_1,M_2,M_3
				  \coordinate (M_1) at ($(B)!.5!(C)$);
				  \draw[] (O) -- (M_1) node[above right] {$M_1$};
				  
				  \coordinate (M_2) at ($(A)!.5!(C)$);
				  %\draw[visible on=<3>] (O) -- (M_2) node[above left] {$M_2$};
				  
				  \coordinate (M_3) at ($(A)!.5!(B)$);
				  %\draw[visible on=<4>] (O) -- (M_3) node[above] {$M_3$};
				  \end{tikzpicture}
				  \end{figure}
			
				  \begin{tcolorbox}[basic] 
				      
				    \smallskip
				    \hypotheses
							      
				    \these
				    \end{tcolorbox}
		
		
		\column{.58\textwidth}\centering
		
		\underline{Résolution} \flushleft
                \underline{$AH\bot BC$:}\medskip
                
			   $\vect{AH}= \vect{AO} + \vect{OH}$ \\[1em]

                         Comme  $\vect{OH} =  3\vect{OG},  \textcolor{blue}{(1.)}$ \\[1em]	
                         et	 $\vect{OG} = \frac{1}{3}\ \vect{OA} + \frac{1}{3}\ \vect{OB} + \frac{1}{3}\ \vect{OC}$, \textcolor{blue}{(2.)}\\[1em]
                         On a
                  $ \vect{OH}  =  \vect{OA} + \vect{OB} + \vect{OC}$. 
                           
        
                
              

                	\begin{align*}
			    \rightarrow \vect{AH}=\vect{AO} + \vect{OH} =\ & \vect{OB}+\vect{OC}, \\[0.3em]
							     =\ & 2\ \vect{OM_1}. \text{ ($M_1$ milieu de $[BC]$)}
        \end{align*}
			\flushleft
		
		
		\heart La droite passant par le centre d'un cercle et le milieu d'une de ses cordes est perpendiculaire à cette corde. \\ \medskip
		
		$AH \parallel OM_1$ et $OM_1\bot BC$ $\rightarrow AH\bot BC$. \\ \medskip
		$AH$ est la hauteur issue de $A$.

	   \end{columns} 
	   \notedir{%
	   .1 Prouver thèse.
	   .2 $H$ orthocentre.
	   .3 Élément de théorie.
	   .4 Orthocentre est intersection de 2 hauteurs..
	   .3 Résolution..
           .4 Décomposer $\vect{AH}$ avec $\vect{OH}$ vu hypothèses..
           .5 $\vect{OH}$ est 3x $\vect{OG}$ (hypothèse 1)..
	   .4 $\vect{OG}$ par rapport aux sommets du $\Delta$..
           .5 $\vect{OH}$ par rapport aux sommets du $\Delta$..
           .4 $\vect{AH}$ en fonction des sommets et ensuite du milieu de $[BC]$..
	   .4 $AH$ hauteur issue de $A$ car.
	   .5 $AH$ parallèle à $OM_1$ $\rightarrow AH\bot BC$..
            .6 Droite passant par centre et milieu corde est\\ \hspace{5mm}$\bot$ corde..
	   .5 $AH\bot BC$ et passe par $A$ $\rightarrow$ $AH$ hauteur..
	   }
    }
    \frame{ 
	  % résolution ex1
	  \begin{columns}[t]
		\column{.46\textwidth}\centering 
		

			\underline{Dessin}\\
				  \vspace{-15mm}
				  \begin{figure}[h]
				  \begin{tikzpicture}[scale=0.8]
			          %projection ($(X)!(B')!(B)$)
			          %nommer chemin 'name path
			          %intersection \path [name intersections={of=d and gb,name=G}];
			          %animation  \draw[visible on=<1>] 
				  %           \draw[visible on=<{2,4}>]
				  %angle arc[radius = 6mm, start angle= 180, end angle= 225] node [below left,pos=0.3]{$\alpha$}
				  %angle \pic [draw,"$\alpha$", angle eccentricity=1.5] {angle = A'--A--B};
				  %perpendiculaire ($(A')!3cm!-90:(A)$)
				  %cercle par point \node [draw] at (A) [circle through=(B)] {};
				  
				  %CERCLE et triangle
				  \coordinate[label=below:$O$] (O) at (0,0);
				  \coordinate[label=above right:$B$] (B) at (70:2);
				  \coordinate[label=above right:$C$] (C) at (10:2);
				  \coordinate[label=above left:$A$] (A) at (170:2);
				  \draw[name path =cercle] (O) circle (2);
				  \draw (A) -- (B)--(C) --cycle;
				  \fill (O) circle (0.025);
				  
				  %G
				  \coordinate[label=above right:$G$] (G) at ($.333*(A)+.333*(B)+.333*(C)$);
				  \fill (G) circle (0.025);
				  %H
				  \path[name path=BH] (B) -- +($2*(B)-2*(A)!(B)!(C)$);
				  \path[name path=CH] (C) -- +($2*(A)!(C)!(B)-2*(C)$);
				  \path [name intersections={of=BH and CH,by=H}];
				  \coordinate[label=above right:$H$] () at (H);
				  \fill(H) circle (0.025);
				  %M_1,M_2,M_3
				  \coordinate (M_1) at ($(B)!.5!(C)$);
				  \draw (O) -- (M_1) node[above right] {$M_1$};
				  
				  \coordinate (M_2) at ($(A)!.5!(C)$);
				  \draw (O) -- (M_2) node[above] {$M_2$};
				  
				  %\coordinate (M_3) at ($(A)!.5!(B)$);
				  %\draw (O) -- (M_3) node[above] {$M_3$};
				
				  \end{tikzpicture}
				  \end{figure}
			
				  \begin{tcolorbox}[basic] 
				      
				    \smallskip
				    \hypotheses
							      
				    \these
				    \end{tcolorbox}
		
		
		\column{.58\textwidth}\centering
		
		\underline{Résolution}\\ \flushleft
		De la même façon,
                
		\begin{align*}
			    \vect{BH}= \vect{BO} + \vect{OH} =\ & \vect{OA}+\vect{OC}, \\[0.3em]
							     =\ & 2\ \vect{OM_2}. \text{ ($M_2$ milieu de $[AC]$)}
			    \end{align*} 
			\flushleft
		
		
		\heart La droite passant par le centre d'un cercle et le milieu d'une de ses cordes est perpendiculaire à cette corde. \\ \medskip
		
		$BH \parallel OM_2$ et $OM_2\bot AC$ $\rightarrow BH\bot AC$. \\ \medskip
		$BH$ est la hauteur issue de $B$. \\ \bigskip \medskip
		
		
		$H\in AH$ et $H\in BH$ $\rightarrow H$ est orthocentre.\\ \hfill $\qed$
	
		\bigskip
		
		
		%\centering\noindent\rule{2cm}{0.4pt}
              \end{columns}
                 \notedir{%
	   .1 Prouver thèse.
	   .2 $H$ orthocentre.
	   .3 Élément de théorie.
	   .4 Orthocentre est intersection de 2 hauteurs..
	   .3 Résolution..
           .4 Décomposer $\vect{BH}$ avec $\vect{OH}$ vu hypothèses..
           .5 $\vect{OH}$ par rapport aux sommets du $\Delta$..
           .4 $\vect{BH}$ en fonction des sommets et ensuite du milieu de $[AC]$..
	   .4 $BH$ hauteur issue de $B$ car.
	   .5 $BH$ parallèle à $OM_2$ $\rightarrow BH\bot AC$..
            .6 Droite passant par centre et milieu corde est\\ \hspace{5mm}$\bot$ corde..
            .5 $BH\bot AC$ et passe par $B$ $\rightarrow$ $BH$ hauteur..
            .3 $H$ appartient aux hauteurs $AH,BH$ $\rightarrow$ $H$ orthocentre..
          }
    }
   \frame{\enonce
           \vfill\center
	  \underline{Résumé}\\ \flushleft
          \begin{itemize}
          \item L'orthocentre est l'intersection des hauteurs d'un $\Delta$,
	  \item Montrer qu'un point est l'orthocentre d'un $\Delta$ = montrer que les droites passant par ce point et chacun des sommets sont des hauteurs,
            \item Montrer qu'une droite passant par un sommet est une hauteur = montrer qu'elle est $\bot$ au côté opposé,
            \item Si $M$ est le milieu de $[AB]$ alors \\[0.3em]$\vect{OM}=\frac{1}{2}\vect{OA}+\frac{1}{2}\vect{OB} \rightarrow \vect{OA}+\vect{OB}= 2\vect{OM}$.
            
            \end{itemize}
            }
\end{document}

%%% Local Variables:
%%% mode: latex
%%% TeX-master: t
%%% End:
