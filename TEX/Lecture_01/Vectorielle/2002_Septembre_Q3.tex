\documentclass[10pt]{beamer}
\usepackage[french]{babel}
\usepackage[utf8]{inputenc}
\usepackage{pgfpages}
\usepackage{dirtree}
\setbeamertemplate{note page}[plain]
\AtEndNote{\vfill \begin{center} mm:hh \end{center}}
\newcommand{\notedir}[1] {
  \note{\dirtree{#1}}}
\def \ion {$^{\circ}$ }
\usepackage{tcolorbox}
\usepackage{tikz}
\usepackage{tikz-3dplot}
\usetikzlibrary{intersections,calc,,angles,quotes,through}
\usepackage{amsmath}
\usepackage{graphicx}
\usepackage{cases}
\def \heart {\textcolor{blue}{$\heartsuit$} }
\def \C {\mathcal{C}}
\def \orthog {\underline{\perp}}
\def\arcos{\operatorname{arcos}}
\def \deg {^{\circ}}

\newcommand{\vect}[1] {
  \overrightarrow{#1}}

\tcbset{%
	basic/.style={colframe=black,
		      colback=white,
		      top= 0mm,
		      bottom = 2mm,
		      boxsep=0mm
		      }
}
\tikzset{
    invisible/.style={opacity=0},
    visible on/.style={alt={#1{}{invisible}}},
    alt/.code args={<#1>#2#3}{%
      \alt<#1>{\pgfkeysalso{#2}}{\pgfkeysalso{#3}} % \pgfkeysalso doesn't change the path
    },
  }

  \def\enonce{\frametitle{Q3 Septembre 2002.}
	  
	  Soit $ABC$ un triangle dont les angles sont aïgus. Le cercle de diamètre $AB$ coupe
	  la hauteur issue de $C$ en des points $X$ et $Y$ ; le cercle de diamètre $AC$ coupe la
	  hauteur issue de $B$ en des points $Z$ et $T$.
	  \begin{enumerate}
            \renewcommand{\theenumi}{\alph{enumi})}
	   \item Montrer que $|\vect{AX}|^2=\vect{AB}\cdot\vect{AC}$.
	   \item En déduire que les points $X, Y, Z$ et $T$ sont sur un même cercle, dont on
		 déterminera le centre.
	  \end{enumerate}}

        \def\hypotheses{ \underline{Hypothèses}
          \begin{enumerate}
                      \item $CX,BZ$ sont des hauteurs,
		      \item $AB,AC$ sont des diamètres.
		      \end{enumerate}
                    }

                    \def\these{\underline{Thèse} 
		      \begin{enumerate}
                        \renewcommand{\theenumi}{\alph{enumi})}
                      \item $|\vect{AX}|^2=\vect{AB}\cdot\vect{AC}$,
                      \item $X,Y,Z,T$ cocycliques.\\ + centre du cercle.
		      \end{enumerate}}

                    \def\dessin{
                      }
    
\begin{document}  
    \beamertemplatenavigationsymbolsempty
    \setlength{\abovedisplayskip}{0pt}
    \setlength{\belowdisplayskip}{0pt}
    \frame{
	  \enonce
	  
	  \vfill
	  
	  \pause
	  % hypothèses et thèse
	  \begin{tcolorbox}[basic] 
	      \begin{columns}[t]
		 
		 \column{.5\textwidth}\centering
		      
		     \hypotheses

		  
		  \column{.5\textwidth}\centering
		      
		      \these

		
	      \end{columns}
	  \end{tcolorbox}
	   \notedir{%
	.1 Énoncé.
	.2 Hypothèses (non visibles sur le dessin)..
	.2 Thèse..
	.2 Grand dessin.. 
	}
    }

    \frame{ 
	  \begin{columns}[t]
		\column{.5\textwidth}\centering 	

			\underline{Dessin}\\
				  \vspace{-10mm}
				  \begin{figure}[h]
				  \begin{tikzpicture}[scale=0.74]
			          %projection ($(X)!(B')!(B)$)
			          %nommer chemin 'name path
			          %intersection \path [name intersections={of=d and gb,by=G}];
			          %animation  \draw[visible on=<1>] 
				  %           \draw[visible on=<{2,4}>]
				  %angle arc[radius = 6mm, start angle= 180, end angle= 225] node [below left,pos=0.3]{$\alpha$}
				  %angle \pic [draw,"$\alpha$", angle eccentricity=1.5] {angle = A'--A--B};
				  %perpendiculaire ($(A')!3cm!-90:(A)$)
				  
				   %TRIANGLE ABC
				  \coordinate[label=below left:$C$] (C) at (-1.5,0);
				  \coordinate[label=below right:$B$] (B) at (3,0);
				  \coordinate[label=above:$A$] (A) at (0,2.5);
				  \draw (A) -- (B) -- (C) -- cycle;			
				  %CERCLES
				  \node [draw,name path=C_AB] at ($(A)!0.5!(B)$) [circle through=(A)] {};
				  \node [draw,name path=C_AC] at ($(A)!0.5!(C)$) [circle through=(A)] {};
				  %T,Z
				  \path [name path=BZ] (B) -- +($2*(A)!(B)!(C)-2*(B)$);
				  \path [name intersections={of=BZ and C_AC,name=k}];				
				  \coordinate[label=below left:$T$](T) at (k-1);
				  \coordinate[label= above left:$Z$](Z) at (k-2);
				  \fill (T) circle (0.065);
				  \draw (B) -- (Z);
				  %X,Y
				  \path [name path=CY] (C) -- +($2*(A)!(C)!(B)-2*(C)$);
				  \path [name intersections={of=CY and C_AB,name=l}];				
				  \coordinate[label=above:$X$](X) at (l-1);
				  \coordinate[label= below right:$Y$](Y) at (l-2);
				  \fill (Y) circle (0.065);
				  \draw (C) -- (X);
				  \end{tikzpicture}
				  \end{figure}
			
		
		
                                  \column{.6\textwidth}\centering
                                  \vspace{-2mm}
                                   \begin{tcolorbox}[basic] 
				      \centering
				    \medskip
				    \hypotheses
                                    	\medskip
				    \these
				    \end{tcolorbox}

                                    
                                  \end{columns}\bigskip
                \centering     \vspace{-2mm}      
		\underline{Résolution}\\ \flushleft
                  	\heart Le produit scalaire est commutatif et distributif. $\qedhere$
                        \vspace{-0.001mm}
                        \vspace{1mm}
		\begin{align*}
                  |\vect{AX}|^2=&\ \vect{AX}\cdot\vect{AX} \\
                             =& \ (\vect{AB}+\vect{BX})\cdot(\vect{AC}+\vect{CX}), \\
                  =& \ \vect{AB}\cdot\vect{AC} + \vect{AB}\cdot\vect{CX} + \vect{BX}\cdot\vect{AC} + \vect{BX}\cdot\vect{CX}, \\
                  =& \  \vect{AB}\cdot\vect{AC} + \vect{BX}\cdot\vect{AC} + \vect{BX}\cdot\vect{CX}, \text{ (\textcolor{blue}{1.})} \\
                   =& \  \vect{AB}\cdot\vect{AC} + \vect{BX}\cdot\vect{AX} \\
			     =& \ \vect{AB}\cdot\vect{AC}. \text{ (\textcolor{blue}{2.})} \hspace{50mm} \qed(a)
		\end{align*}
	
		
	       

   
    \notedir{%
	   .1 Prouver thèse.
	   .2 $|\vect{AX}|^2=\vect{AB}\cdot\vect{AC}$.
	   .3 Élément de théorie.
	   .4 Produit scalaire de 2 vecteurs $\bot$ est nul..
	   .3 Résolution..
	   .4 Partir d'un membre et faire apparaître vecteurs de\\ \hspace{5mm} l'autre membre..
           .4 Le carré de la longueur d'un vecteur est le produit scalaire de celui-ci par lui-même.. 
	   .4 Décomposer les $\vect{AX}$ pour faire apparaître $\vect{AB}$ et $\vect{AC}$..
	   .5 $\vect{AB}\cdot\vect{CX}=0$ car $CX$ hauteur..
           .5 $\vect{AC} + \vect{CX}=\vect{AX}$ par Chasles..
           .5 $\vect{BX}\cdot\vect{AX}=0$ car $AB$ diamètre..
	   .4 $|\vect{AX}|^2=\vect{AB}\cdot\vect{AC}$..
	   }
    }
	\frame{ 
	  % résolution ex1
	  \begin{columns}[t]
		\column{.5\textwidth}\centering 
		

			\underline{Dessin}\\
				  \vspace{-10mm}
				  \begin{figure}[h]
				  \begin{tikzpicture}[scale=0.74]
			          %projection ($(X)!(B')!(B)$)
			          %nommer chemin 'name path
			          %intersection \path [name intersections={of=d and gb,by=G}];
			          %animation  \draw[visible on=<1>] 
				  %           \draw[visible on=<{2,4}>]
				  %angle arc[radius = 6mm, start angle= 180, end angle= 225] node [below left,pos=0.3]{$\alpha$}
				  %angle \pic [draw,"$\alpha$", angle eccentricity=1.5] {angle = A'--A--B};
				  %perpendiculaire ($(A')!3cm!-90:(A)$)
				  
				   %TRIANGLE ABC
				  \coordinate[label=below left:$C$] (C) at (-1.5,0);
				  \coordinate[label=below right:$B$] (B) at (3,0);
				  \coordinate[label=above:$A$] (A) at (0,2.5);
				  \draw (A) -- (B) -- (C) -- cycle;			
				  %CERCLES
				  \node [draw,name path=C_AB] at ($(A)!0.5!(B)$) [circle through=(A)] {};
				  \node [draw,name path=C_AC] at ($(A)!0.5!(C)$) [circle through=(A)] {};
				  %T,Z
				  \path [name path=BZ] (B) -- +($2*(A)!(B)!(C)-2*(B)$);
				  \path [name intersections={of=BZ and C_AC,name=k}];				
				  \coordinate[label=below left:$T$](T) at (k-1);
				  \coordinate[label= above left:$Z$](Z) at (k-2);
				  \fill (T) circle (0.065);
				  \draw (B) -- (Z);
				  %X,Y
				  \path [name path=CY] (C) -- +($2*(A)!(C)!(B)-2*(C)$);
				  \path [name intersections={of=CY and C_AB,name=l}];				
				  \coordinate[label=above:$X$](X) at (l-1);
				  \coordinate[label= below right:$Y$](Y) at (l-2);
				  \fill (Y) circle (0.065);
				  \draw (C) -- (X);
				  \end{tikzpicture}
				  \end{figure}
			
				  
		
		
                                 \column{.6\textwidth}\centering
                                  \vspace{-2mm}
                                   \begin{tcolorbox}[basic] 
				      \centering
				    \medskip
				    \hypotheses
                                    	\medskip
				    \these
				    \end{tcolorbox}

                                    
                                  \end{columns}\bigskip
                \centering           
		\underline{Résolution}\\ \flushleft
		\vspace{-3mm}
		\begin{align*}
		  |\vect{AZ}|^2=&\ \vect{AZ}\cdot\vect{AZ} \\
                             =& \ (\vect{AB}+\vect{BZ})\cdot(\vect{AC}+\vect{CZ}), \\
                  =& \ \vect{AB}\cdot\vect{AC} + \vect{AB}\cdot\vect{CZ} + \vect{BZ}\cdot\vect{AC} + \vect{BZ}\cdot\vect{CZ}, \\
                  =& \  \vect{AB}\cdot\vect{AC} + \vect{AB}\cdot\vect{CZ} + \vect{BZ}\cdot\vect{CZ}, \text{ (\textcolor{blue}{1.})} \\
                   =& \  \vect{AB}\cdot\vect{AC} + \vect{AZ}\cdot\vect{CZ} \\
			     =& \ \vect{AB}\cdot\vect{AC}. \text{ (\textcolor{blue}{2.})}
		\end{align*}
	  
    \notedir{%
	   .1 Prouver thèse.
	   .2 $X,Y,Z,T$ cocycliques.\\ + centre du cercle.
	   .3 Élément de théorie.
	   .4 Des points cocycliques sont à égal distance d'un point (le centre du cercle)..
	   .3 Résolution..
	   .4 Partir d'un membre et faire apparaître vecteurs de\\ \hspace{5mm} l'autre membre..
           .4 Le carré de la longueur d'un vecteur est le produit scalaire de celui-ci par lui-même.. 
	   .4 Décomposer les $\vect{AZ}$ pour faire apparaître $\vect{AB}$ et $\vect{AC}$..
	   .5 $\vect{BZ}\cdot\vect{AC}=0$ car $BZ$ hauteur..
           .5 $\vect{AB} + \vect{BZ}=\vect{AZ}$ par Chasles..
           .5 $\vect{AZ}\cdot\vect{CZ}=0$ car $AC$ diamètre..
	   .4 $|\vect{AZ}|^2=\vect{AB}\cdot\vect{AC}$..
	   }
    }  

    	\frame{ 
	  % résolution ex1
	  \begin{columns}[t]
		\column{.5\textwidth}\centering 
		

			\underline{Dessin}\\
				  \vspace{-10mm}
				  \begin{figure}[h]
				  \begin{tikzpicture}[scale=0.74]
			          %projection ($(X)!(B')!(B)$)
			          %nommer chemin 'name path
			          %intersection \path [name intersections={of=d and gb,by=G}];
			          %animation  \draw[visible on=<1>] 
				  %           \draw[visible on=<{2,4}>]
				  %angle arc[radius = 6mm, start angle= 180, end angle= 225] node [below left,pos=0.3]{$\alpha$}
				  %angle \pic [draw,"$\alpha$", angle eccentricity=1.5] {angle = A'--A--B};
				  %perpendiculaire ($(A')!3cm!-90:(A)$)
				  
				   %TRIANGLE ABC
				  \coordinate[label=below left:$C$] (C) at (-1.5,0);
				  \coordinate[label=below right:$B$] (B) at (3,0);
				  \coordinate[label=above:$A$] (A) at (0,2.5);
				  \draw (A) -- (B) -- (C) -- cycle;			
				  %CERCLES
				  \node [draw,name path=C_AB] at ($(A)!0.5!(B)$) [circle through=(A)] {};
				  \node [draw,name path=C_AC] at ($(A)!0.5!(C)$) [circle through=(A)] {};
				  %T,Z
				  \path [name path=BZ] (B) -- +($2*(A)!(B)!(C)-2*(B)$);
				  \path [name intersections={of=BZ and C_AC,name=k}];				
				  \coordinate[label=below left:$T$](T) at (k-1);
				  \coordinate[label= above left:$Z$](Z) at (k-2);
				  \fill (T) circle (0.065);
				  \draw (B) -- (Z);
				  %X,Y
				  \path [name path=CY] (C) -- +($2*(A)!(C)!(B)-2*(C)$);
				  \path [name intersections={of=CY and C_AB,name=l}];				
				  \coordinate[label=above:$X$](X) at (l-1);
				  \coordinate[label= below right:$Y$](Y) at (l-2);
				  \fill (Y) circle (0.065);
				  \draw (C) -- (X);
				  \end{tikzpicture}
				  \end{figure}
			
				  
		
		
                                  \column{.6\textwidth}\centering
                                  \vspace{-2mm}
                                   \begin{tcolorbox}[basic] 
				      \centering
				    \medskip
				    \hypotheses
                                    	\medskip
				    \these
				    \end{tcolorbox}

                                    
                                  \end{columns}\bigskip
                \centering           
		\underline{Résolution}\\ \flushleft
		\vspace{-3mm}
                De façon similaire, on montre que \\
                $$ |\vect{AY}|^2 = \vect{AB}\cdot\vect{AC} \text{ et } |\vect{AT}|^2 = \vect{AB}\cdot\vect{AC}$$
                Donc,
		$$|\vect{AX}|^2=|\vect{AY}|^2=|\vect{AZ}|^2=|\vect{AT}|^2$$ 
		$\rightarrow X,Y,Z,T$ cocycliques et le centre du cercle est $A$. \hfill $\qed(b)$
	  
    \notedir{%
	   .1 Prouver thèse.
	   .2 $X,Y,Z,T$ cocycliques + centre du cercle.
	   .3 Élément de théorie.
	   .4 Des points cocycliques sont à égal distance d'un point (le centre du cercle)..
	   .3 Résolution..
	   .4 Partir d'un membre et faire apparaître vecteurs de\\ \hspace{5mm} l'autre membre..
           .4 Le carré de la longueur d'un vecteur est le produit scalaire de celui-ci par lui-même.. 
	   .4 Décomposer les $\vect{AZ}$ pour faire apparaître $\vect{AB}$ et $\vect{AC}$..
	   .5 $\vect{BZ}\cdot\vect{AC}=0$ car $BZ$ hauteur..
           .5 $\vect{AB} + \vect{BZ}=\vect{AZ}$ par Chasles..
           .5 $\vect{AZ}\cdot\vect{CZ}=0$ car $AC$ diamètre..
	   .4 $|\vect{AZ}|^2=\vect{AB}\cdot\vect{AC}$..
           .4 Par mêmes calculs, $ |\vect{AY}|^2 = \vect{AB}\cdot\vect{AC} = |\vect{AT}|^2$.
           .4 Comme $|\vect{AX}|^2,\ |\vect{AY}|^2,\ |\vect{AZ}|^2 \text{ et }|\vect{AT}|^2$ sont tous égaux à\\ \hspace{5mm} un même nombre, ils sont égaux entre eux.
           .5 Donc les 4 points $X,\ Y,\ Z\text{ et } T$ sont à la même distance de $A$..
	   }
         }

         \frame{\enonce
           \vfill\center
	  \underline{Résumé}\\ \flushleft
          \begin{itemize}
	  \item Montrer l'égalité d'une expression vectorielle = décomposer vecteurs d'un membre avec les vecteurs du second membre.
	  \item $|\vect{AB}|^2=\vect{AB}\cdot\vect{AB}$.
          \item  Le produit scalaire est commutatif et distributif.
          \item Le produit scalaire de 2 vecteurs $\perp$ est nul.
          \item Montrer que des points sont cocycliques = montrer qu'ils sont à la même distance d'un point (le centre du cercle).
          \end{itemize}

           }
\end{document}

%%% Local Variables:
%%% mode: latex
%%% TeX-master: t
%%% End:
