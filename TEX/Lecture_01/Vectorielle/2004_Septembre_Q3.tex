\documentclass[10pt]{beamer}

\usepackage[utf8]{inputenc}
\usepackage{pgfpages}
\usepackage[french]{babel}
\usepackage{dirtree}
\setbeamertemplate{note page}[plain]
\AtEndNote{\vfill \begin{center} mm:hh \end{center}}
\newcommand{\notedir}[1] {
  \note{\dirtree{#1}}}
\def \ion {$^{\circ}$ }
\usepackage{tcolorbox}
\usepackage{tikz}
\usepackage{tikz-3dplot}
\usetikzlibrary{intersections,calc,,angles,quotes,through}
\usepackage{amsmath}
\usepackage{graphicx}
\usepackage{cases}
\def \heart {\textcolor{blue}{$\heartsuit$} }
\def \C {\mathcal{C}}
\def \orthog {\underline{\perp}}
\def\arcos{\operatorname{arcos}}
\def\cos{\operatorname{cos}}
\def \deg {^{\circ}}

\newcommand{\vect}[1] {
  \overrightarrow{#1}}
  
\tcbset{%
	basic/.style={colframe=black,
		      colback=white,
		      top= 0mm,
		      bottom = 2mm,
		      boxsep=0mm
		      }
}
\tikzset{
    invisible/.style={opacity=0},
    visible on/.style={alt={#1{}{invisible}}},
    alt/.code args={<#1>#2#3}{%
      \alt<#1>{\pgfkeysalso{#2}}{\pgfkeysalso{#3}} % \pgfkeysalso doesn't change the path
    },
  }

    
\begin{document}  
    \beamertemplatenavigationsymbolsempty
    \setlength{\abovedisplayskip}{0pt}
    \setlength{\belowdisplayskip}{0pt}

    \def\enonce{  \frametitle{Q3 Septembre 2004.}
	  %\renewcommand{\theenumi}{\alph{enumi})}
	  On considère un triangle $ABC$ (non isocèle) du plan. On construit sur le côté
	  $BC$ et dans le plan du triangle $ABC$ deux triangles équilatéraux $BDC$ et $BD'C$.
	  Démontrer la relation \\ \medskip
	  $|\vect{AD}|^2 + |\vect{AD'}|^2 = |\vect{AB}|^2 + |\vect{AC}|^2 + |\vect{BC}|^2$.
        }
        \def\hypotheses{ \underline{Hypothèses} 
		      \begin{enumerate}
		      \item $\Delta BDC, \Delta BD'C$ équilatéraux.
		      \end{enumerate}
                    }
                    \def\these{  \underline{Thèse} \\
		      \smallskip
		      $|\vect{AD}|^2 + |\vect{AD'}|^2$ \\ \smallskip$= |\vect{AB}|^2 + |\vect{AC}|^2 + |\vect{BC}|^2$.
                      }
    \frame{
	  
	\enonce

	  \vfill
	  
	  \pause
	  % hypothèses et thèse
	  \begin{tcolorbox}[basic] 
	      \begin{columns}[t]
		 
		 \column{.52\textwidth}\centering
		      
		     \hypotheses

		  
		  \column{.5\textwidth}\centering
		      
		    \these
		
	      \end{columns}
	  \end{tcolorbox}
	  \notedir{%
	.1 Énoncé.
	.2 Hypothèses (non visibles sur le dessin)..
	.2 Thèse..
	.2 Grand dessin.. 
	}
    }

    \frame{ 
	  % résolution ex1
	  \begin{columns}[t]
		\column{.5\textwidth}\centering 
		

			\underline{Dessin}\\
			
				  \begin{figure}[h]
				  \begin{tikzpicture}[scale=0.48]
			          %projection ($(X)!(B')!(B)$)
			          %nommer chemin 'name path
			          %intersection \path [name intersections={of=d and gb,by=G}];
			          %animation  \draw[visible on=<1>] 
				  %           \draw[visible on=<{2,4}>]
				  %angle arc[radius = 6mm, start angle= 180, end angle= 225] node [below left,pos=0.3]{$\alpha$}
				  %angle \pic [draw,"$\alpha$", angle eccentricity=1.5] {angle = A'--A--B};
				  %perpendiculaire ($(A')!3cm!-90:(A)$)
				  
				  %TRIANGLE ABC
				  \coordinate[label= above left:$A$](A) at (0.8,1.2);
				  \coordinate[label=left:$B$](B) at (-2,0);
				  \coordinate[label=right:$C$](C) at (2,0);
				  \draw (A) -- (B) -- (C) -- cycle;
				  
				  %D
				  \draw (B) -- +(60:4) coordinate[label=above:$D$](D) -- (C);
				  %D'
				  \draw (B) -- +(-60:4) coordinate[label=below:$D'$](D') -- (C);
				  
				  %ANGLES
				  \pic [draw,"$\alpha '$", angle eccentricity=1.5,above] {angle = C--B--D};
				  \pic [draw,"$\alpha$", angle eccentricity=1.5] {angle = B--C--D'};
				  \end{tikzpicture}
				  \end{figure}
			
				  \begin{tcolorbox}[basic] 
				      
				    \smallskip
				    \hypotheses
							      
				  \these
				    \end{tcolorbox}
		
		
		\column{.5\textwidth}\centering
		
		\underline{Résolution}\\ \flushleft
		\begin{align*}
		 \vect{AD}=& \vect{AB} + \vect{BD}, \\
		 \vect{AD'}=& \vect{AC} + \vect{CD'}. 		 
		\end{align*}
		
		 $\begin{cases}\alpha = \alpha ' = 60\deg \rightarrow BD \parallel CD', \\ 
		       |BD|=|CD'|.
		       \end{cases} \textcolor{blue}{(1.)}
$ 		 
	\vspace{-2mm}	
\flushright$\rightarrow \vect{CD'}=-\vect{BD}$ \\[0.5em]\medskip
		
	\begin{align*}
		 \vect{AD}=& \vect{AB} + \vect{BD}, \\
		 \vect{AD'}=& \vect{AC} - \vect{BD}. 		 
		\end{align*}		
		
		\begin{align*}
                  &|\vect{AD}|^2 + |\vect{AD'}|^2 \\[0.3em]
                  &=\vect{AD}\cdot\vect{AD} + \vect{AD'}\cdot\vect{AD'} \\[0.3em]
                  &=(\vect{AB} + \vect{BD})\cdot(\vect{AB} + \vect{BD}) \\ &\hspace{20mm}+ (\vect{AC} - \vect{BD})\cdot(\vect{AC} - \vect{BD}) \\[0.3em]
		 &= |\vect{AB}|^2 + |\vect{AC}|^2 + 2|\vect{BD}|^2  \\
		          &\phantom{{}=1}+ 2\vect{BD}\cdot(\vect{CA} + \vect{AB}), \\[0.3em]
		 &= |\vect{AB}|^2 + |\vect{AC}|^2 +2\vect{BD}\cdot (\vect{CB} + \vect{BD}), \\[0.3em]
		 &= |\vect{AB}|^2 + |\vect{AC}|^2 +2\vect{BD}\cdot\vect{CD},
		\end{align*}

	   \end{columns}
   \notedir{%
	   .1 Prouver thèse.
	   .2 $|\vect{AD}|^2 + |\vect{AD'}|^2= |\vect{AB}|^2 + |\vect{AC}|^2 + |\vect{BC}|^2$.
	   .3 Élément de théorie.
	   .4 Décomposer les vecteurs d'un membre de la thèse pour faire apparaître ceux du 2\ieme\ membre et tenir compte hypothèses..
	   .3 Résolution à partir membre de gauche..
	   .4 Décomposition des vecteurs pour faire apparaître $\vect{AB}$ et $\vect{AC}$.
	   .4 $\vect{BD}=\vect{D'C}$ car.
	   .5 $\Delta$ équilatéraux $\rightarrow\alpha = \alpha ' = 60\deg \rightarrow BD \parallel CD'$..
	   .5 $\Delta$ équilatéraux $\rightarrow |BD|=|CD'|$..
	   .4 Injecter décomposition en notant que $\vect{BD}=\vect{D'C}$..
	   }
    }

        \frame{ 
	  % résolution ex1
	  \begin{columns}[t]
		\column{.5\textwidth}\centering 
		

			\underline{Dessin}\\
			
				  \begin{figure}[h]
				  \begin{tikzpicture}[scale=0.48]
			          %projection ($(X)!(B')!(B)$)
			          %nommer chemin 'name path
			          %intersection \path [name intersections={of=d and gb,by=G}];
			          %animation  \draw[visible on=<1>] 
				  %           \draw[visible on=<{2,4}>]
				  %angle arc[radius = 6mm, start angle= 180, end angle= 225] node [below left,pos=0.3]{$\alpha$}
				  %angle \pic [draw,"$\alpha$", angle eccentricity=1.5] {angle = A'--A--B};
				  %perpendiculaire ($(A')!3cm!-90:(A)$)
				  
				  %TRIANGLE ABC
				  \coordinate[label= above left:$A$](A) at (0.8,1.2);
				  \coordinate[label=left:$B$](B) at (-2,0);
				  \coordinate[label=right:$C$](C) at (2,0);
				  \draw (A) -- (B) -- (C) -- cycle;
				  
				  %D
				  \draw (B) -- +(60:4) coordinate[label=above:$D$](D) -- (C);
				  %D'
				  \draw (B) -- +(-60:4) coordinate[label=below:$D'$](D') -- (C);
				  
				  %ANGLES
				  \pic [draw,"$\alpha '$", angle eccentricity=1.5,above] {angle = C--B--D};
				  \pic [draw,"$\alpha$", angle eccentricity=1.5] {angle = B--C--D'};
				  \end{tikzpicture}
				  \end{figure}
			
				  \begin{tcolorbox}[basic] 
				      
				    \smallskip
				    \hypotheses
							      
				  \these
				    \end{tcolorbox}
		
		
		\column{.60\textwidth}\centering
		
		\underline{Résolution}\\ \flushleft
		
		\begin{align*}
		 &= \vect{AB}\cdot\vect{AB} + \vect{AC}\cdot\vect{AC}   \\
                 &\phantom{{}=1}+ 2\vect{BD}\cdot(\vect{CA} + \vect{AB} + \vect{BD}), \\[0.3em]
                 &= \vect{AB}\cdot\vect{AB} + \vect{AC}\cdot\vect{AC}+ 2\vect{BD}\cdot\vect{CD}, \\[0.3em]
                 &= \vect{AB}\cdot\vect{AB} + \vect{AC}\cdot\vect{AC}+ 2(-\vect{DB}\cdot-\vect{DC}), \\[0.3em]
                 &= \vect{AB}\cdot\vect{AB} + \vect{AC}\cdot\vect{AC}+ 2\vect{DB}\cdot\vect{DC}, \\[0.3em]
                 &=  |\vect{AB}||\vect{AB}|\cos(0) + |\vect{AC}||\vect{AC}|\cos(0)\\ &\hspace{30mm}+ 2|\vect{DB}||\vect{DC}|\cos(\underbrace{\widehat{BDC}}_{60^\circ}), \\[0.3em]
                 &=  |\vect{AB}||\vect{AB}| + |\vect{AC}||\vect{AC}|+|\vect{DB}||\vect{DC}|, \\[0.3em]
                   &= |\vect{AB}|^2 + |\vect{AC}|^2 + |\vect{BC}|^2.\ \textcolor{blue}{(1.)}
                \end{align*}
		
                \hfill \qed

	   \end{columns}
   \notedir{%
	   .1 Prouver thèse.
	   .2 $|\vect{AD}|^2 + |\vect{AD'}|^2= |\vect{AB}|^2 + |\vect{AC}|^2 + |\vect{BC}|^2$.
	   .3 Élément de théorie.
	   .4 Décomposer les vecteurs d'un membre de la thèse pour faire apparaître ceux du 2\ieme\ membre et tenir compte hypothèses..
	   .3 Résolution à partir membre de gauche..
	   .4 Décomposition des vecteurs pour faire apparaître $\vect{AB}$ et $\vect{AC}$.
	   .4 $\vect{BD}=\vect{D'C}$ car.
	   .5 $\Delta$ équilatéraux $\rightarrow\alpha = \alpha ' = 60\deg \rightarrow BD \parallel CD'$..
	   .5 $\Delta$ équilatéraux $\rightarrow |BD|=|CD'|$..
	   .4 Réexprimer décomposition en notant que $\vect{BD}=\vect{D'C}$..
           .4 Évaluer membre de gauche avec les décompositions..
           .5 Produit scalaire commutatif et distributif..
	   .4 Mise en évidence de $\vect{BD}$ et $\vect{CA} + \vect{AB} + \vect{BD}=\vect{CD}$ par \\ \hspace{5mm} Chasles..
           .4 Exprimer le dernier produit scalaire avec des vecteurs de même origine..
           .4 Exprimer les produits scalaires en termes de longueurs et angles..
	   }
    }

      \frame{\enonce
           \vfill\center
	  \underline{Résumé}\\ \flushleft
          \begin{itemize}
	  \item Montrer l'égalité d'une expression vectorielle = décomposer vecteurs d'un membre avec les vecteurs du second membre.
	  \item $|\vect{AB}|^2=\vect{AB}\cdot\vect{AB}$.
          \item  $\vect{AB}\cdot\vect{AC}=|\vect{AB}||\vect{AC}|\cos(\widehat{BAC})$
            
            \end{itemize}
            }
\end{document}

%%% Local Variables:
%%% mode: latex
%%% TeX-master: t
%%% End:
