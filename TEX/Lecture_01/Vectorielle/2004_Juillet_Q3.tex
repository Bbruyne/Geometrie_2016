\documentclass[10pt]{beamer}

\usepackage[utf8]{inputenc}
\usepackage{pgfpages}
\usepackage[french]{babel}
\usepackage{dirtree}
\setbeamertemplate{note page}[plain]
\AtEndNote{\vfill \begin{center} mm:hh \end{center}}
\newcommand{\notedir}[1] {
  \note{\dirtree{#1}}}
\def \ion {$^{\circ}$ }
\usepackage{tcolorbox}
\usepackage{tikz}
\usepackage{tikz-3dplot}
\usetikzlibrary{intersections,calc,,angles,quotes,through}
\usepackage{amsmath}
\usepackage{graphicx}
\usepackage{cases}
\def \heart {\textcolor{blue}{$\heartsuit$} }
\def \C {\mathcal{C}}
\def \orthog {\underline{\perp}}
\def\arcos{\operatorname{arcos}}
\def \deg {^{\circ}}

\newcommand{\vect}[1] {
  \overrightarrow{#1}}

\tcbset{%
	basic/.style={colframe=black,
		      colback=white,
		      top= 0mm,
		      bottom = 2mm,
		      boxsep=0mm
		      }
}
\tikzset{
    invisible/.style={opacity=0},
    visible on/.style={alt={#1{}{invisible}}},
    alt/.code args={<#1>#2#3}{%
      \alt<#1>{\pgfkeysalso{#2}}{\pgfkeysalso{#3}} % \pgfkeysalso doesn't change the path
    },
  }

  \def\enonce{ \frametitle{Q3 Juillet 2004.}
	  %\renewcommand{\theenumi}{\alph{enumi})}
	  Soit $ABC$ un triangle tel que l’angle $\widehat{A}$ soit obtus. On note $E$ la projection
	  orthogonale de $B$ sur $AC$ et $F$ la projection orthogonale de $C$ sur $AB$. \\ \smallskip
	  Démontrer que $|\vect{BC}|^2 = |\vect{BF}||\vect{AB}| + |\vect{AC}||\vect{CE}|$.  }
        \def\hypotheses{  \underline{Hypothèses} 
		      \begin{enumerate}
		      \item $FC \bot BA$,
		      \item $BE \bot AC$.
		      \end{enumerate}
        }

        \def\these{\underline{Thèse} \\
		      \smallskip
		      $|\vect{BC}|^2 = |\vect{BF}||\vect{AB}| + |\vect{AC}||\vect{CE}|$.  }
\begin{document}  
    \beamertemplatenavigationsymbolsempty
    \setlength{\abovedisplayskip}{0pt}
    \setlength{\belowdisplayskip}{0pt}
    \frame{
	  
	 \enonce
	  \vfill
	  
	  \pause
	  % hypothèses et thèse
	  \begin{tcolorbox}[basic] 
	      \begin{columns}[t]
		 
		 \column{.5\textwidth}\centering
		      
		    
                 \hypotheses
		  
		  \column{.5\textwidth}\centering
		      
		      \these
		
	      \end{columns}
	  \end{tcolorbox}
	   \notedir{%
	.1 Énoncé.
	.2 Hypothèses (non visibles sur le dessin)..
	.2 Thèse..
	.2 Grand dessin.. 
	}
    }

    \frame{ 
	  % résolution ex1
	  \begin{columns}[t]
		\column{.5\textwidth}\centering 
		

			\underline{Dessin}\\
			
				  \begin{figure}[h]
				  \begin{tikzpicture}[scale=0.8]
			          %projection ($(X)!(B')!(B)$)
			          %nommer chemin 'name path
			          %intersection \path [name intersections={of=d and gb,name=G}];
			          %animation  \draw[visible on=<1>] 
				  %           \draw[visible on=<{2,4}>]
				  %angle arc[radius = 6mm, start angle= 180, end angle= 225] node [below left,pos=0.3]{$\alpha$}
				  %angle \pic [draw,"$\alpha$", angle eccentricity=1.5] {angle = A'--A--B};
				  %perpendiculaire ($(A')!3cm!-90:(A)$)
				  %cercle par point \node [draw] at (A) [circle through=(B)] {};
				  
				   %TRIANGLE ABC
				  \coordinate[label=below left:$C$] (C) at (-1.5,0);
				  \coordinate[label=below right:$B$] (B) at (3,0);
				  \coordinate[label=above:$A$] (A) at (0,1.5);
				  \draw (A) -- (B) -- (C) -- cycle;		
				  \fill (A) circle (0.08);
				  %E,F
				  \draw [name path=BE] (B) -- ($(A)!(B)!(C)$) coordinate[label=above left:$E$] (E) -- (A);
				  \draw [name path=CF] (C) -- ($(A)!(C)!(B)$) coordinate[label=above left:$F$] (F) -- (A);
				  \end{tikzpicture}
				  \end{figure}
			
				  \begin{tcolorbox}[basic] 
				      
				    \smallskip
				    \hypotheses
							      
				    \these
				    \end{tcolorbox}
		
		
		\column{.52\textwidth}\centering
		
		\underline{Résolution}\\ \flushleft
		
		\heart Le produit scalaire de 2 vecteurs perpendiculaires est nul.\\
		
		\begin{align*}
                  |\vect{BC}|^2 =& \vect{BC}\cdot\vect{BC}, \\[0.5em]
                  =& (\vect{BF} + \vect{FC})\cdot (\vect{BA} + \vect{AC}),\\[0.5em]
                  =& \vect{BF}\cdot\vect{BA} + \vect{BF}\cdot\vect{AC} + \vect{FC}\cdot\vect{BA} + \vect{FC}\cdot\vect{AC},\\[0.5em]
                  =&  \vect{BF}\cdot\vect{BA} + \vect{BF}\cdot\vect{AC} + \vect{FC}\cdot\vect{AC}, (\textcolor{blue}{1.})\\[0.5em]
                  =&  \vect{BF}\cdot\vect{BA} + (\vect{BF}+\vect{FC})\cdot\vect{AC},\\[0.5em]
                  =&  \vect{BF}\cdot\vect{BA} + \vect{BC}\cdot\vect{AC},\\[0.5em]
                  =&  \vect{BF}\cdot\vect{BA} + (\vect{BE} + \vect{EC})\cdot\vect{AC},\\[0.5em]
                  =&  \vect{BF}\cdot\vect{BA} + \vect{EC}\cdot\vect{AC}, (\textcolor{blue}{2.})\\[0.5em]
		\end{align*}
		
		%\centering\noindent\rule{2cm}{0.4pt}
	        %\hfill $\qed$

   
	   \end{columns}
    \notedir{%
	   .1 Prouver thèse.
	   .2 $|\vect{BC}|^2 = |\vect{AB}||\vect{BF}| + |\vect{AC}||\vect{CE}|$.  
	   .3 Élément de théorie.
	   .4 Décomposer les vecteurs d'un membre de la thèse avec ceux du 2\ieme\ membre et tenir compte des hypothèses..
	   .4 Décomposition de $\vect{BC}$  pour faire apparaître $\vect{BF}$ et\\ \hspace{5mm} $\vect{AB}$.. 
	   .5 $\vect{FC}\cdot\vect{BA}=0$ car $F$ projection orthogonale de $B$..
           .4 $\vect{BF}+\vect{FC}=\vect{BC}$ par Chasles.. 
           .4 Décomposition de $\vect{BC}$  pour faire apparaître $\vect{CE}$.. 
	   .5 $\vect{BE}\cdot\vect{AC}=0$ car $E$ projection orthogonale de $B$..
	   }
    }

        \frame{ 
	  % résolution ex1
	  \begin{columns}[t]
		\column{.5\textwidth}\centering 
		

			\underline{Dessin}\\
			
				  \begin{figure}[h]
				  \begin{tikzpicture}[scale=0.8]
			          %projection ($(X)!(B')!(B)$)
			          %nommer chemin 'name path
			          %intersection \path [name intersections={of=d and gb,name=G}];
			          %animation  \draw[visible on=<1>] 
				  %           \draw[visible on=<{2,4}>]
				  %angle arc[radius = 6mm, start angle= 180, end angle= 225] node [below left,pos=0.3]{$\alpha$}
				  %angle \pic [draw,"$\alpha$", angle eccentricity=1.5] {angle = A'--A--B};
				  %perpendiculaire ($(A')!3cm!-90:(A)$)
				  %cercle par point \node [draw] at (A) [circle through=(B)] {};
				  
				   %TRIANGLE ABC
				  \coordinate[label=below left:$C$] (C) at (-1.5,0);
				  \coordinate[label=below right:$B$] (B) at (3,0);
				  \coordinate[label=above:$A$] (A) at (0,1.5);
				  \draw (A) -- (B) -- (C) -- cycle;		
				  \fill (A) circle (0.08);
				  %E,F
				  \draw [name path=BE] (B) -- ($(A)!(B)!(C)$) coordinate[label=above left:$E$] (E) -- (A);
				  \draw [name path=CF] (C) -- ($(A)!(C)!(B)$) coordinate[label=above left:$F$] (F) -- (A);
				  \end{tikzpicture}
				  \end{figure}
			
				  \begin{tcolorbox}[basic] 
				      
				    \smallskip
				    \hypotheses
							      
				    \these
				    \end{tcolorbox}
		
		
		\column{.5\textwidth}\centering
		
		\underline{Résolution}\\ \flushleft
		
	
		\begin{align*} 
                  &= |\vect{BF}||\vect{BA}|\cos(0) + |\vect{EC}||\vect{AC}|\cos(0)\\[0.5em]
                                 &=  |\vect{BF}||\vect{BA}| + |\vect{EC}||\vect{AC}|\\[0.5em]
                                 &=  |\vect{BF}||\vect{AB}| + |\vect{CE}||\vect{AC}|\\[0.5em]
		\end{align*}
	        \hfill $\qed$

   
	   \end{columns}
       \notedir{%
	   .1 Prouver thèse.
	   .2 $|\vect{BC}|^2 = |\vect{AB}||\vect{BF}| + |\vect{AC}||\vect{CE}|$.  
	   .3 Élément de théorie.
	   .4 Décomposer les vecteurs d'un membre de la thèse avec ceux du 2\ieme\ membre et tenir compte des hypothèses..
	   .4 Décomposition de $\vect{BC}$  pour faire apparaître $\vect{BF}$ et\\ \hspace{5mm} $\vect{AB}$.. 
	   .5 $\vect{FC}\cdot\vect{BA}=0$ car $F$ projection orthogonale de $B$..
           .4 $\vect{BF}+\vect{FC}=\vect{BC}$ par Chasles.. 
           .4 Décomposition de $\vect{BC}$  pour faire apparaître $\vect{CE}$.. 
	   .5 $\vect{BE}\cdot\vect{AC}=0$ car $E$ projection orthogonale de $B$..
           .4 Exprimer les produits scalaires en termes de \\ \hspace{5mm} longueurs et angles..
           .5 L'angle entre $\vect{BF}$ et $\vect{BA}$ est nul..
           .5 L'angle entre $\vect{EC}$ et $\vect{AC}$ est nul..
           .4 La longeur d'un vecteur est indépendante de la \\ \hspace{5mm} direction de celui-ci..
           .5 $|\vect{BA}|=|\vect{AB}|$ et $|\vect{EC}|=|\vect{CE}|$..
	   }
    }
    \frame{\enonce
           \vfill\center
	  \underline{Résumé}\\ \flushleft
          \begin{itemize}
	  \item Montrer l'égalité d'une expression vectorielle = décomposer vecteurs d'un membre avec les vecteurs du second membre.
	  \item $|\vect{AB}|^2=\vect{AB}\cdot\vect{AB}$.
          \item   Le produit scalaire de 2 vecteurs perpendiculaires est nul.
          \item  La longeur d'un vecteur est indépendante de la direction de celui-ci.
            
            \end{itemize}
            }
\end{document}

%%% Local Variables:
%%% mode: latex
%%% TeX-master: t
%%% End:
