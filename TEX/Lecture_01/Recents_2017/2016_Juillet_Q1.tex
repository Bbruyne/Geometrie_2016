\documentclass[10pt]{beamer}
\usepackage[frenchb]{babel}
\usepackage[utf8]{inputenc}
\usepackage{pgfpages}
\usepackage{dirtree}
\setbeamertemplate{note page}[plain]
\setbeameroption{show notes on second screen =left}
\AtEndNote{\vfill \begin{center} mm:hh \end{center}}
\newcommand{\notedir}[1] {
  \note{\dirtree{#1}}}
\def \ion {$^{\circ}$ }
\usepackage{tcolorbox}
\usepackage{tikz}
\usetikzlibrary{intersections,calc}
\usepackage{amsmath}
\usepackage{graphicx}
\def \heart {\textcolor{blue}{$\heartsuit$} }
\def \C {$\mathcal{C}$}



\tcbset{%
	basic/.style={colframe=black,
		      colback=white,
		      top= 0mm,
		      bottom = 2mm,
		      boxsep=0mm
		      }
                    }
                    \def\enonce{ \frametitle{Q1 Juillet 2016.} 
	  
	  
	  Par un point $P$ intérieur à un cercle $\mathcal{C}$ de centre $O$ et de rayon $r$, on
	  mène deux droites perpendiculaires $d_1$ et $d_2$ . On note $A_1$ un des points
	  d’intersection de $d_1$ avec $\mathcal{C}$, et $A_2$ un des points d’intersection de $d_2$
	  avec $\mathcal{C}$. Le milieu de la corde $[A_1 A_2 ]$ est noté $M$. Démontrer l’égalité
	  $$ |OM|^2 + |PM|^2 = r^2,$$
	  où $|XY|$ dénote la longueur du segment $[XY]$.
                    }
                    \def\hypotheses{ 
		      \underline{Hypothèses} 
		      \begin{enumerate}
		      \item $d_1 \bot d_2$,
		      \item $|A_1M|=|MA_2|$.
		      \end{enumerate}

                    }
                    \def\these{ \underline{Thèse} \\
		      \smallskip
		      $ |OM|^2 + |PM|^2 = r^2$.
                      }

\begin{document}  
    \beamertemplatenavigationsymbolsempty
    
    \frame{ % énoncé ex1
	  
	 \enonce
	  
	  \vfill
	  
	  \pause
	  % hypothèses et thèse
	  \begin{tcolorbox}[basic] 
	      \begin{columns}[t]
		 
		 \column{.5\textwidth}\centering
		     
		  \hypotheses
		  \column{.5\textwidth}\centering
		      
		     \these
		
	      \end{columns}
	  \end{tcolorbox}
	\notedir{%
	.1 Énoncé.
	.2 Choix : Géométrie synthétique.
	.3 Égalité de longueurs et présence de $\Delta$..
	.4 Hypothèses (non visibles sur le dessin)..
	.4 Thèse..
	.4 Grand dessin.. 
	}
    
    }
    
    \frame{ % résolution ex1
	  \begin{columns}[t]
		\column{.5\textwidth}\centering 
		

			\underline{Dessin}\\
			
				  \begin{figure}[h]
				  \begin{tikzpicture}[scale=0.8]
					%\draw[help lines] (-3,-3) grid (3,3); 
					
					%CERCLE
					\coordinate[label=below left:$O$] (O) at (0,0);
					\fill (O) circle (0.05);
					\draw[name path=cercle] (O) circle (2cm);
					\node[below left] (C) at (-110:2) {$\mathcal{C}$};
					
					% P
					\coordinate[label=left:$P$] (P) at (160:1.5);
					
					%D_1 et D_2
					\draw[name path=d_1] (P) -- +(3,-3) node[right]() {$d_1$};
					\draw (P) -- +(-1,1);
					\draw[name path=d_2] (P) -- +(2,2) node [right]() {$d_2$};
					\draw (P) -- +(-1,-1);
							
					%A_1,A_2 et M 
					\path [name intersections={of=d_1 and cercle,by=A_1}];
					\node [below] at (A_1) {$A_1$};
					\path [name intersections={of=d_2 and cercle,by=A_2}];
					\node [above left] at (A_2) {$A_2$}; 
					\draw (A_1) -- (A_2) coordinate[pos=0.5,label=right:$M$](M);
					
					
					%segments [PM],[OM],[OA_1]
					\draw (P) -- (M);
					\draw[thick] (O) -- node[xshift=-1mm,above left]{$r$} (A_1);
					\draw[thick] (O) -- (M);
                                        \draw[thick] (M) -- (A_1);
				  \end{tikzpicture}
				  \end{figure}
			
				  \begin{tcolorbox}[basic] 
				      
				    \smallskip
				   \hypotheses
							      
				    \these
				    \end{tcolorbox}
		
		
		\column{.5\textwidth}\centering
		
		\underline{Résolution}\\

                	\begin{itemize} 
		\item[$\heartsuit$]La droite passant par le centre d'un cercle et le milieu d'une de ses cordes est perpendiculaire à cette dernière.
		\end{itemize}

                $\rightarrow \Delta OMA_1$ est rectangle. \\[0.5em]
                $\rightarrow |OM|^2 + |MA_1|^2 = r^2$.
	
		
		\begin{itemize} 
		\item[$\heartsuit$]La longueur de la médiane issue de l'angle droit d'un $\Delta$ rectangle vaut la moitié de celle de l'hypothénuse.
		\end{itemize}

                \begin{enumerate}
		 \item $\Delta A_1A_2P$ rectangle,
		 \item $[PM]$ médiane du $\Delta A_1A_2P$. 
		\end{enumerate}
                
		$\rightarrow |PM| = |MA_1|$ \\
		
	\medskip
Donc,
		\hfill $|OM|^2 + |PM|^2 = r^2$ \hfill $\qed$

   
	   \end{columns}
	   \notedir{%
	   .1 Prouver la thèse.
	   .2 Élément de théorie.
	   .3 Pythagore..
	   .4 Égalité de longueurs..
	   .4 Somme de carrés..
	   .2 Résolution.
            .3 Théorème de la médiatrice d'une corde..
            .4 Application à $OM$ et $A_1A_2$ $\rightarrow \Delta OMA_1$ rectangle..
            .5 Pythagore dans $ \Delta OMA_1$ ressemble à thèse..
	   .3 Théorème de la médiane issue de l'angle droit..
	   .4 Application à $\Delta A_1A_2P \rightarrow |PM| = |MA_1|$..
	   .3 Thèse vérifiée..
	   }   
         }
                    \frame{ 
	  \enonce

	  \vfill\center
	  \underline{Résumé}\\ \flushleft
          \begin{itemize}
	  \item  La droite passant par le centre d'un cercle et le milieu d'une de ses cordes est perpendiculaire à cette dernière.
          \item La longueur de la médiane issue de l'angle droit d'un $\Delta$ rectangle vaut la moitié de celle de l'hypothénuse.
            \item Somme de carrés de longueur dans thèse $\rightarrow$ Pythagore ?
          \end{itemize}
    }
	 
 
  
	  
	  
\end{document}

%%% Local Variables:
%%% mode: latex
%%% TeX-master: t
%%% End:
