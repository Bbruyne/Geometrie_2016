\documentclass[10pt]{beamer}
\usepackage[frenchb]{babel}
\usepackage[utf8]{inputenc}
\usepackage{pgfpages}
\usepackage{dirtree}
\setbeamertemplate{note page}[plain]
\setbeameroption{show notes on second screen =left}
\AtEndNote{\vfill \begin{center} mm:hh \end{center}}
\newcommand{\notedir}[1] {
  \note{\dirtree{#1}}}
\def \ion {$^{\circ}$ }
\usepackage{tcolorbox}
\usepackage{tikz}
\usetikzlibrary{intersections,calc}
\usepackage{amsmath}
\usepackage{graphicx}
\def \heart {\textcolor{blue}{$\heartsuit$} }
\def \C {$\mathcal{C}$}



\tcbset{%
	basic/.style={colframe=black,
		      colback=white,
		      top= 0mm,
		      bottom = 2mm,
		      boxsep=0mm
		      }
}

\def\enonce{	\frametitle{Q2 Juillet 2016.} 
	  
	  On donne une droite $d$ tangente à un cercle $\mathcal{C}$, et on considère le lieu
	  des points dont la distance à $\mathcal{C}$ est égale à la distance à $d$. On demande
	  de caractériser ce lieu à l’aide d’équations cartésiennes, de préciser la
	  nature de celui-ci, et de le représenter graphiquement.
}
\def\procede{ \underline{Procédé} \\
		      \begin{itemize}
		       \item Poser un repère orthonormé,
		       \item Exprimer la condition pour qu'un point $P$ appartienne au lieu. 		            
		      \end{itemize}
		   	 $\mathcal{D}(P,d) = \mathcal{D}(P,\mathcal{C})$.
}

\begin{document}  
 \beamertemplatenavigationsymbolsempty
 
  \frame{ % énoncé ex2
	
\enonce
	  	  
	  \vfill
	  
	  \pause	  
	  % Conditions
	  \begin{tcolorbox}[basic] 
		
		      \centering \smallskip
		     	  \procede
	  \end{tcolorbox}
	  	  
	 \notedir{%
	.1 Énoncé.
	.2 Choix : géométrie analytique.
	.3 Énoncé évoque équations cartésiennes..
	.4 Procédé : reformuler ce qu'il faut déterminer..
	.5 Lieu = ensemble de points vérifiant une condition..
	}  
	}
	
  \frame{ %Résolution
	
	\begin{columns}[t]
		\column{.5\textwidth}\centering 
		

			\underline{Dessin}\\
			
				  \begin{figure}[h]
				  \begin{tikzpicture}[scale=0.67]
					%\draw[help lines] (-3,-3) grid (3,3); 
					
					%AXES
					\draw[->] (0,-3) -- (0,3) coordinate[label=below right:$y$]();
					\draw[->] (-3,0) -- (3,0) coordinate[label=above left:$x$]();
					%CERCLE									
					\draw (0,0) coordinate[label=below left:$O$] circle (1.5cm);
					\node[below left] (C) at (110:2) {$\mathcal{C}$};
					\coordinate[label=above:$r$] () at (1.5/2,0);
					%DROITE D
					\draw (-3,-1.5) coordinate[label= below right:$d$]() -- (3,-1.5); 
					%Point P
					\coordinate[label=above:$P$] (P) at (-30:2);
                                        % Projection P sur d
                                        \coordinate[label=below:$P_{\bot}$] (P') at (1.73,-1.5);
                                        
                                        % Projection P sur C
                                        \coordinate[label=above left:$P_\mathcal{C}$] (Pc) at (-30:1.5);
                                        
					\fill (P) circle (0.05cm);
                                        \fill (P') circle (0.05cm);
                                         \fill (Pc) circle (0.05cm);

                                         \draw[dotted] (P) -- (P');
                                           \draw[dotted] (P) -- (Pc);

				
				  \end{tikzpicture}
				  \end{figure}
			\begin{tcolorbox}[basic]
			\centering \smallskip
		      \procede	  
			\end{tcolorbox}
			
		\column{.5\textwidth}\centering
		
		\underline{Résolution}\\
		Soit un repère orthonormé $R=(O,X,Y)$ avec :
		\begin{itemize}
		 \item[$\bullet$] $P=(x,y)$,
		 \item[$\bullet$] $\mathcal{C} \equiv x^2 + y ^2 = r^2$,
		 \item[$\bullet$] $d \equiv y = -r$.
		\end{itemize} \medskip
		
		\begin{enumerate}\item
		  \begin{align*}\mathcal{D}(P,d) =& \mathcal{D}(P,P_\bot)\\
                    =& \sqrt{(y-(-r))^2 + (x-x)^2}\\
                    =& |y+r|\end{align*}
                \item
                  \begin{align*}\mathcal{D}(P,\mathcal{C})=&\mathcal{D}(P,P_\mathcal{C}) \\=&|\mathcal{D}(O,P) -\mathcal{D}(O,P_\mathcal{C})|  \\=& |\sqrt{(x-0)^2+(y-0)^2}-r|  \\=& |\sqrt{x^2+y^2}-r|\end{align*}
		\end{enumerate}
		
		
	\end{columns}
	\notedir{%
	.1 Déterminer le lieu.
	.2 Représenter les éléments dans repère.
	.3 $P$: point vérifiant la condition du lieu..
	.3 $\mathcal{C}$: cercle de rayon r centré sur $O$..
	.3 $d$: tangente au cercle..	
	.2 Résolution.
	.3 Condition: $\mathcal{D}(P,d) = \mathcal{D}(P,\mathcal{C})$.
	.4 $\mathcal{D}$ point-droite = $\mathcal{D}$ entre point et sa projection.. 
	.4 $\mathcal{D}$ point-cercle = $\mathcal{D}$ point et centre - rayon..
	}
	}
  
  \frame{%Résolution
	 \centering
	 \bigskip
	 $|y+r| = |\sqrt{x^2+y^2}-r|$ \medskip
	 \begin{columns}[t]
	  \column{.5\textwidth}\centering
	  Pour :
	  \begin{itemize}
	   \item $P$ au-dessus de $d$ et $P$ à l'extérieur du cercle, ou
	   \item $P$ en dessous de $d$ et $P$ à l'intérieur du cercle (imp.).
	  \end{itemize}
	  \begin{align*}
	  y+r &= \sqrt{x^2+y^2}-r \\
	  y&=\dfrac{x^2}{4r}-r\quad (r\neq0).\\
	  \end{align*}
	  

	  \column{.5\textwidth}\centering
	  Pour :
	  \begin{itemize}
	   \item $P$ au-dessus de $d$ et $P$ à l'intérieur du cercle, ou
	   \item $P$ en dessous de $d$ et $P$ à l'extérieur du cercle.
	  \end{itemize}
	  \begin{align*}
	  y+r =& -(\sqrt{x^2+y^2}-r) \\
	  y   =& -\sqrt{x^2+y^2} \quad (\rightarrow y\leq0) \\
	  x   =& 0\quad  (\text{ et } y\leq0). \\
	  \end{align*}
	 \end{columns}
	
	\vspace{-5mm}
	\underline{Dessin}\\
			
				  \begin{figure}[h]
				  \begin{tikzpicture}[scale=0.58]
					%\draw[help lines] (-3,-3) grid (3,3); 
					
					%AXES
					\draw[->] (0,-3) -- (0,3) coordinate[label=below right:$y$]();
					\draw[->] (-5,0) -- (5,0) coordinate[label=above left:$x$]();
					%CERCLE									
					\draw (0,0) circle (1.5cm);
					\node[below left] (C) at (110:2) {$\mathcal{C}$};
					\coordinate[label=above:$r$] () at (1.5/2,0);
					%DROITE D
					\draw (-3,-1.5) coordinate[label= below right:$d$]() -- (3,-1.5); 
					%LIEU
					\fill (0,0) circle (0.1);
					\draw[very thick] (0,0) -- (0,-3);
					\draw[very thick] (-5,2.66667) parabola bend(0,-1.5) (5,2.66667);
					
				  \end{tikzpicture}
				  \end{figure}

	
                                  \notedir{%
	.1 Égaliser les deux distances $\rightarrow$ étude du signe..
	.2 Les membres sont de même signe.
	.3 Tous les 2 positifs.
	.4 $P$ au dessus de $d$ et à l'ext.~cercle..
	.3 Tous les 2 négatifs.
	.4 $P$ en dessous de $d$ et à l'int.~cercle..
	.3 On trouve parabole comprise dans les régions ci-dessus..
	.2 Les membres sont de signes différents.
	.3 1\ier{} positif et 2\ieme{} négatif.
	.4 $P$ au dessus de $d$ et à l'int.~cercle..
	.3 1\ier{} négatif et 2\ieme{} positif.
	.4 $P$ en dessus de $d$ et à l'ext.~cercle..
	.3 On trouve $x=0$ à restreindre dans les régions ci-dessus..
	.2 Dessin du lieu..
	}
      }

                          \frame{ 
	  \enonce

	  \vfill\center
	  \underline{Résumé}\\ \flushleft
          \begin{itemize}
	  \item La distance entre un point et une courbe est la distance entre ce point et sa projection sur la courbe.
          \item $\sqrt{x^2}=|x|$.
          \item Étudier le signe lorsqu'on retire valeur absolue d'une égalité.
            \item Avant l'élévation au carré d'une égalité, noter les contraintes de signe imposées par la racine.
          \end{itemize}
    }
\end{document}

%%% Local Variables:
%%% mode: latex
%%% TeX-master: t
%%% End:
