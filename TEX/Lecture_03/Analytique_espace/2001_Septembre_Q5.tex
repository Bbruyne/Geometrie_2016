\documentclass[10pt]{beamer}

\usepackage[utf8]{inputenc}
\usepackage{pgfpages}
\usepackage[french]{babel}
\usepackage{dirtree}
\setbeamertemplate{note page}[plain]
\setbeameroption{show notes on second screen =left}
\AtEndNote{\vfill \begin{center} mm:hh \end{center}}
\newcommand{\notedir}[1] {
  \note{\dirtree{#1}}}
\def \ion {$^{\circ}$ }
\usepackage{tcolorbox}
\usepackage{tikz}
\usepackage{tikz-3dplot}
\usetikzlibrary{intersections,calc,,angles,quotes,through}
\usepackage{amsmath}
\usepackage{graphicx}
\usepackage{cases}
\def \heart {\textcolor{blue}{$\heartsuit$} }
\def \C {\mathcal{C}}
\def \orthog {\underline{\perp}}
\def\arcos{\operatorname{arcos}}
\def \deg {^{\circ}}

\newcommand{\vect}[1] {
  \overrightarrow{#1}}

\tcbset{%
	basic/.style={colframe=black,
		      colback=white,
		      top= 0mm,
		      bottom = 2mm,
		      boxsep=0mm
		      }
}
\tikzset{
    invisible/.style={opacity=0},
    visible on/.style={alt={#1{}{invisible}}},
    alt/.code args={<#1>#2#3}{%
      \alt<#1>{\pgfkeysalso{#2}}{\pgfkeysalso{#3}} % \pgfkeysalso doesn't change the path
    },
  }

    
\begin{document}  
    \beamertemplatenavigationsymbolsempty
    \setlength{\abovedisplayskip}{0pt}
    \setlength{\belowdisplayskip}{0pt}
    \frame{
	  
	  \frametitle{Q5 Septembre 2001.}
	  \renewcommand{\theenumi}{\alph{enumi})}
	  \begin{enumerate}
	   \item Quelles sont les coordonnées du pied $Q$ de la perpendiculaire abaissée du point
		 $P (\alpha, \alpha + 1, \alpha - 1)$ sur le plan $\pi$ d’équation $2x + \alpha y + \alpha z + \alpha^3 + 4 = 0$ ?
	   \item Montrer que ces pieds, lorsque $\alpha$ parcourt $\mathbb{R}$, sont tous situés sur une même
		 droite $d$, dont on déterminera des équations.
	  \end{enumerate}

	  \vfill
	  
	  \pause
	  
	   \begin{tcolorbox}[basic] \smallskip
	     \centering\underline{Procédé}
	     \begin{columns}[c]
	     \column{.5\textwidth}\centering 
	     \renewcommand{\theenumi}{\alph{enumi})}
	     \begin{enumerate}
	      \item $PQ \equiv \begin{cases}
	                        \vect{n_\pi}, \\
	                        \ni P.
	                       \end{cases}$ \\ \medskip
	            $\phantom{P}Q = PQ \cap \pi$.


	     \end{enumerate}
	     \column{.5\textwidth}\centering 
	     \vspace{-7mm}
	     \begin{enumerate}
	      \item[b).] Éliminer le paramètre $\alpha$.

	     \end{enumerate}

	  \end{columns}
	  \end{tcolorbox}
	  \notedir{%
	.1 Énoncé.
	.2 Procédé : reformuler ce qu'il faut déterminer..
	}
    }

    \frame{ 
	  % résolution ex1

		
				  \begin{tcolorbox}[basic] 
				      
				    \smallskip
				    \centering\underline{Procédé}
				    \begin{columns}[c]
				    \column{.5\textwidth}\centering 
				    \renewcommand{\theenumi}{\alph{enumi})}
				    \begin{enumerate}
				      \item $PQ \equiv \begin{cases}
							\vect{n_\pi}, \\
							\ni P.
						      \end{cases}$ \\ \medskip
					    $\phantom{P}Q = PQ \cap \pi$.


				    \end{enumerate}
				    \column{.5\textwidth}\centering 
				    \vspace{-7mm}
				    \begin{enumerate}
				      \item[b).] Éliminer le paramètre $\alpha$.

				    \end{enumerate}

				    \end{columns}
				    \end{tcolorbox}
		
		
		
		\centering\underline{Résolution}\\ \medskip
		
		$\pi \equiv 2x + \alpha y + \alpha z + \alpha^3 + 4 = 0 \qquad \rightarrow \qquad \vect{n_\pi}=(2,\alpha,\alpha)$. \bigskip\flushleft
		\begin{columns}
		
		\column{.5\textwidth}\centering 
		\begin{align*}
		 PQ &\ \equiv \begin{cases}
		            \vect{n_\pi}, \\
			    \ni P(\alpha, \alpha + 1, \alpha - 1).
		           \end{cases} \\
		    &\ \equiv \frac{\alpha(x-\alpha)}{2}=y-\alpha-1=z-\alpha+1.
		\end{align*}\bigskip
		
		\column{.5\textwidth}\centering
		\begin{figure}[h]
				  \begin{tikzpicture}[scale=0.85]
			          %projection ($(X)!(B')!(B)$)
			          %nommer chemin 'name path
			          %intersection \path [name intersections={of=d and gb,by=G}];
			          
			          %\draw[help lines] (-3,-3) grid (3,3);
			          \draw[dotted] (2,0,0) -- (2,0,2) node[above left]{$\pi$} -- (-2,0,2) -- (-2,0,0);
			          \coordinate[label=above:$P$](P) at (0,0.8,1);
			          \fill (P) circle (0.025);
			          \draw[->] (-1.5,0,1) -- (-1.5,0.5,1) node[left]{$\vect{n_\pi}$};
			          \draw (P) -- (0,0,1) node[left]{$Q$};
			          \draw (0,0,1) -- (0,-0.5,1);
			          \draw (0,0,0.85) -- (0,0,1.15) (-0.08,0,1) -- (0.08,0,1);
			          
				  \end{tikzpicture}
				  \end{figure}
		\end{columns}
		
		$Q =  PQ\ \cap \ \pi = \begin{cases}
		                     \frac{\alpha(x-\alpha)}{2}=y-\alpha-1, \\
		                     y-\alpha-1=z-\alpha+1, \\
		                     2x + \alpha y + \alpha z + \alpha^3 + 4 = 0.
		                    \end{cases}
		                    \hspace{-2mm} \rightarrow  Q(-2,\frac{2-\alpha^2}{2},-\frac{2+\alpha^2}{2})$.

		\hfill $\qed(a)$
		%\centering\noindent\rule{2cm}{0.4pt}
	\notedir{%
	.1 Suivre procédé.
	.2 Coordonnées de $Q$.
	.3 Élement de théorie.
	.4 $Q$ intersect\ion plan avec droite $\bot$ plan passant par $P$..
	.3 Résolution.
	.4 $PQ$ déterminée par vecteur normal plan et point $P$.
	.5 Coordonnées vecteur normal..
	.6 Coefficients devant $x,y,z$ de équation du plan.. 
	.5 Son équation $\dfrac{(x-P_{1_x})}{n_{\pi_x}}=\dfrac{(y-P_{1_y})}{n_{\pi_y}}=\dfrac{(z-P_{1_z})}{n_{\pi_z}}$. 
	.4 $Q$ intersection droite $PQ$ et plan $\pi$..
	}
    }
	  
	  
    \frame{ 
	  % résolution ex1

		
				  \begin{tcolorbox}[basic] 
				      
				    \smallskip
				    \centering\underline{Procédé}
				    \begin{columns}[c]
				    \column{.5\textwidth}\centering 
				    \renewcommand{\theenumi}{\alph{enumi})}
				    \begin{enumerate}
				      \item $PQ \equiv \begin{cases}
							\vect{n_\pi}, \\
							\ni P.
						      \end{cases}$ \\ \medskip
					    $\phantom{P}Q = PQ \cap \pi$.


				    \end{enumerate}
				    \column{.5\textwidth}\centering 
				    \vspace{-7mm}
				    \begin{enumerate}
				      \item[b).] Éliminer le paramètre $\alpha$.

				    \end{enumerate}

				    \end{columns}
				    \end{tcolorbox}
		
		
		
		\centering\underline{Résolution}\\ \medskip
		$Q(-2,\frac{2-\alpha^2}{2},-\frac{2+\alpha^2}{2})\qquad \rightarrow \qquad \begin{cases}
											    x = -2, \\
											    y = \frac{2-\alpha^2}{2}, \; \text{ ($\dagger$)}\\
											    z = -\frac{2+\alpha^2}{2}. 
											    \end{cases}$ \bigskip
											    
		$(\dagger)\quad \alpha^2 = -2y + 2  \qquad \rightarrow \qquad d \equiv      \begin{cases}
											    x = -2, \\
											    y = z + 2. 
											    \end{cases}$  \\ \bigskip\flushleft
											    
		$d$ est une droite car intersection de 2 plans. \\ \bigskip
		De $(\dagger)$, si $\alpha \in \mathbb{R}$, $y\le 1$.\\ \medskip
		Lieu est demi-droite $d$ comprise dans région $y\le 1$. \hfill $\qed(b)$
		
		%\centering\noindent\rule{2cm}{0.4pt}
	\notedir{%
	.1 Suivre procédé.
	.2 Lieu des pieds $Q$.
	.3 Élement de théorie.
	.4 Équations paramétriques.
	.5 On fait varier un paramètre en entrée, on reçoit coordonnées des points en sortie..
	.4 Équations cartésiennes.
	.5 On teste les coordonnées d'un point dans des équations pour vérifier son appartenance..
	.3 Résolution.
	.4 On a équations paramétriques de $Q$..
	.5 En faisant varier $\alpha$ dans $\mathbb{R}$, on a coordonnées points..
	.4 Pour montrer que c'est une droite, il faut equat\ion cartésiennes..
	.4 Équations paramétriques vers cartésiennes.	
	.5 Isoler $\alpha$ et observer domaine de variation de $y$..
	.5 Injecter $\alpha$ dans une autre équation pour l'éliminer..
	.4 L'équation du lieu est une demi-droite car intersect\ion 2 plans et $y\ge0$ car $\alpha \in \mathbb{R}$ (dans $(\dagger)$)..
	}
    }
  
\end{document}
