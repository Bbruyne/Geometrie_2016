\documentclass[10pt]{beamer}

\usepackage[utf8]{inputenc}
\usepackage{pgfpages}
\usepackage[french]{babel}
\usepackage{dirtree}
\setbeamertemplate{note page}[plain]
\AtEndNote{\vfill \begin{center} mm:hh \end{center}}
\newcommand{\notedir}[1] {
  \note{\dirtree{#1}}}
\def \ion {$^{\circ}$ }
\usepackage{tcolorbox}
\usepackage{tikz}
\usepackage{tikz-3dplot}
\usetikzlibrary{intersections,calc,,angles,quotes,through}
\usepackage{amsmath}
\usepackage{graphicx}
\usepackage{cases}
\def \heart {\textcolor{blue}{$\heartsuit$} }
\def \C {\mathcal{C}}
\def \orthog {\underline{\perp}}
\def\arcos{\operatorname{arcos}}
\def \deg {^{\circ}}

\newcommand{\vect}[1] {
  \overrightarrow{#1}}

\tcbset{%
	basic/.style={colframe=black,
		      colback=white,
		      top= 0mm,
		      bottom = 2mm,
		      boxsep=0mm
		      }
}
\tikzset{
    invisible/.style={opacity=0},
    visible on/.style={alt={#1{}{invisible}}},
    alt/.code args={<#1>#2#3}{%
      \alt<#1>{\pgfkeysalso{#2}}{\pgfkeysalso{#3}} % \pgfkeysalso doesn't change the path
    },
  }

  \def\enonce{ \frametitle{Q4 Juillet 2011.}
	  Dans un repère orthonormé de l’espace, on considère la droite $d_1$, pas-
	  sant par les points $A$ et $B$ respectivement de coordonnées $(1, 2, 3)$ et
	  $(-1, 0, 2)$ et la droite $d_2$ , passant par les points $C$,$D$ respectivement
	  de coordonnées $(0, 1, 7)$ et $(2, 0, 5)$.
	  \renewcommand{\theenumi}{(\alph{enumi})}
	  \begin{enumerate}
	   \item Déterminer l’équation cartésienne du plan $\Pi$ parallèle à la droite
		 $d_1$ et contenant la droite $d_2$.
	   \item Déterminer la distance entre la droite $d_1$ et le plan $\Pi$.
	   \item Déterminer des équations paramétriques et des équations carté-
		 siennes de la droite $d_3$ passant par $C$ et orthogonale à $d_1$ et $d_2$.
	   \item Déterminer un point $P_1$ de $d_1$ et point $P_2$ de $d_2$ tels que le vecteur
		 joignant $P_1$ à $P_2$ soit orthogonal à $d_1$ et à $d_2$ .
	  \end{enumerate}

    }
    
\begin{document}  
    \beamertemplatenavigationsymbolsempty
    \setlength{\abovedisplayskip}{0pt}
    \setlength{\belowdisplayskip}{0pt}
    
    
    \frame{
	  
	 \enonce
	  
	  \vfill
	  
	  \pause
	  % hypothèses et thèse
	  \begin{tcolorbox}[basic] 
	  \medskip
	  \centering\underline{Inconnues}
	  \medskip
	      \begin{columns}[c]
		 
		 \column{.25\textwidth}\centering
		      
		      
		         $\Pi$     $\begin{cases} 
		                    \overrightarrow{d_1} \\
		                    \supset{d_2}
		                   \end{cases}$
		 \column{.20\textwidth}\centering       
		      $\mathcal{D}(d_1,\Pi)$
		      

		  
		  \column{.25\textwidth}\centering
		      
		          $d_3$	  $\begin{cases} 
		                    \overrightarrow{d_1}\wedge \overrightarrow{d_2}  \\
		                    \ni C
		                   \end{cases}$
		  \column{.30\textwidth}\centering
				   $\begin{cases} 
				    P_1 \in d_1 \\
				    P_2 \in d_2 \\
		                    \overrightarrow{P_1P_2}\ \alpha\ \overrightarrow{d_1}\wedge \overrightarrow{d_2}  \\
		                    
		                   \end{cases}$
		      
		      
		
	      \end{columns}
	  \end{tcolorbox}
	  \notedir{%
	  .1 Énoncé.
	  .2 Procédé : reformuler ce qu'il faut déterminer..
	  .3 Équation de $\Pi$.
	  .4 Parallèle à $d_1$ et inclus $d_2$..
	  .3 Distance entre $d_1$ et $\Pi$..
	  .3 Équation de $d_3$.
	  .4 Perpendiculaire à $d_1$ et à $d_2$..
	  .4 Passe par $C$..
	  .3 Points $P_1,P_2$.
	  .4 $P_1$ est sur $d_1$..
	  .4 $P_2$ est sur $d_2$..
	  .4 $\vect{P_1P_2}$ perpendiculaire à $d_1$ et à $d_2$..
	   }
    }

   \frame{ 			
				  \begin{tcolorbox}[basic] 
				      
				    \medskip
	  \centering\underline{Inconnues}
	  \medskip
	      \begin{columns}[c]
		 
		 \column{.25\textwidth}\centering
		      
		      
		         $\Pi$     $\begin{cases} 
		                    \overrightarrow{d_1} \\
		                    \supset{d_2}
		                   \end{cases}$
		 \column{.20\textwidth}\centering       
		      $\mathcal{D}(d_1,\Pi)$
		      

		  
		  \column{.25\textwidth}\centering
		      
		          $d_3$	  $\begin{cases} 
		                    \overrightarrow{d_1}\wedge \overrightarrow{d_2}  \\
		                    \ni C
		                   \end{cases}$
		  \column{.30\textwidth}\centering
				   $\begin{cases} 
				    P_1 \in d_1 \\
				    P_2 \in d_2 \\
		                    \overrightarrow{P_1P_2}\ \alpha\ \overrightarrow{d_1}\wedge \overrightarrow{d_2}  \\
		                    
		                   \end{cases}$
		      
		      
		
	      \end{columns}
				    \end{tcolorbox}
	
		\centering\underline{Résolution}\\ \flushleft
		\medskip
		\begin{columns}[c]
	
		\column{.5\textwidth}\centering
		\begin{align*}
	        d_1 &\equiv \begin{cases} 
		      A(1,2,3), \\
		      B(-1,0,2).
			    \end{cases} \\ 			    
	        d_1 &\equiv \begin{cases}
		            \vect{AB} = (-2,-2,-1), \\
		            A(1,2,3).
		            \end{cases}      
		\end{align*}
					       
		\column{.5\textwidth}\centering			       
		 \begin{align*}
	        d_2 &\equiv \begin{cases} 
		      C(0,1,7), \\
		      D(2,0,5).
			    \end{cases} \\ 			    
	        d_2 &\equiv \begin{cases}
		            \vect{CD} = (2,-1,-2), \\
		            C(0,1,7).
		            \end{cases}      
		\end{align*}
		\end{columns}
		\bigskip
			
		\begin{align*}
		\Pi \equiv& \begin{cases}
		                C, \\
		                \vect{AB}, \\
		                \vect{CD}.
		               \end{cases} \\
		      \Pi \equiv&\ x-2y+2z-12=0.
		\end{align*} \hfill $\qed(a)$
    \notedir{%
    .1 Équation de $\Pi$.
    .2 Exprimer droites avec point et vecteur..
    .2 $\Pi$ défini par.
    .3 Point $C$..
    .3 Direction de $d_1$..
    .3 Direction de $d_2$..
    .2 Équations cartésiennes obtenues en annulant le déterminant.
    .3 1\iere\ colonne : $x - C_x$,$y - C_y$, $z - C_z$..
    .3 2\ieme\ colonne : coordonnée de $\vect{AB}$..
    .3 3\ieme\ colonne : coordonnée de $\vect{CD}$..
    }
    }
    \frame{ 
	    			
				  \begin{tcolorbox}[basic] 
				      
				    \medskip
	  \centering\underline{Inconnues}
	  \medskip
	      \begin{columns}[c]
		 
		 \column{.25\textwidth}\centering
		      
		      
		         $\Pi$     $\begin{cases} 
		                    \overrightarrow{d_1} \\
		                    \supset{d_2}
		                   \end{cases}$
		 \column{.20\textwidth}\centering       
		      $\mathcal{D}(d_1,\Pi)$
		      

		  
		  \column{.25\textwidth}\centering
		      
		          $d_3$	  $\begin{cases} 
		                    \overrightarrow{d_1}\wedge \overrightarrow{d_2}  \\
		                    \ni C
		                   \end{cases}$
		  \column{.30\textwidth}\centering
				   $\begin{cases} 
				    P_1 \in d_1 \\
				    P_2 \in d_2 \\
		                    \overrightarrow{P_1P_2}\ \alpha\ \overrightarrow{d_1}\wedge \overrightarrow{d_2}  \\
		                    
		                   \end{cases}$
		      
		      
		
	      \end{columns}
				    \end{tcolorbox}
		\centering 
		$\mathcal{D}(d_1,\Pi)=\mathcal{D}(A,\Pi)\ \forall A\in d_1.\quad \text{ car $d_1\parallel\Pi$}$ \\ \vspace{8mm}
		\begin{columns}[c]	
		\column{.5\textwidth}\centering
		\begin{figure}[h]
				  \begin{tikzpicture}[scale=0.85]
			          %projection ($(X)!(B')!(B)$)
			          %nommer chemin 'name path
			          %intersection \path [name intersections={of=d and gb,by=G}];
			          
			          %\draw[help lines] (-3,-3) grid (3,3);
			          \draw[dotted] (2,0,0) -- (2,0,2) node[above left]{$\Pi$} -- (-2,0,2) -- (-2,0,0);
			          \coordinate[label=above:$A$](A) at (0,0.8,1);
			          \fill (A) circle (0.025);
				  \draw (-1.5,0.8,1) -- (1.5,0.8,1)node[right]{$d_1$};
			          \draw (A) -- node[right]{$\mathcal{D}(A,\Pi)$}(0,0,1);
			           \draw[->] (-1.5,0,1) -- (-1.5,0.5,1) node[left]{$\vect{n_\Pi}$};
			          \draw (0,0,0.85) -- (0,0,1.15) (-0.08,0,1) -- (0.08,0,1);
			          
				  \end{tikzpicture}
				  \end{figure}
		\column{.52\textwidth}\centering
		\heart Distance point $A$ à un plan $\Pi$ de vecteur normal $\vect{n_{\Pi}}$ est donnée par: \\ \medskip
		$$\mathcal{D}(A,\Pi) = \frac{|n_{\Pi_x}A_x + n_{\Pi_y}A_y + n_{\Pi_z}A_z + d|}{|\vect{n_{\Pi}}|}$$
		\end{columns} \vspace{8mm}
		De (a), $\ \Pi \equiv x-2y+2z-12=0 \quad \rightarrow \quad \vect{n_{\Pi}}=(1,-2,2)$. \\ \medskip

                \hspace{65mm} $| \vect{n_{\Pi}}|=\sqrt{1^2+(-2)^2+2^2}=3.$\\ \medskip
		$$\mathcal{D}(A,\Pi) = 3.$$ \hfill $\qed(b)$
	\notedir{%
	.1 Distance entre $d_1$ et $\Pi$.
	.2 C'est distance entre point de $d_1$ et $\Pi$ car $d_1\parallel\Pi$..
	.3 Formule distance point-plan.
	.4 Mémo : valeur absolue de équation du plan en remplaçant x,y,z par coordonnées du point divisé par longueur vecteur normal..
	.2 Coordonnées du vecteur normal sont coefficient de l'équation du plan..
	.2 Distance vaut 3..
	}
    }
  \frame{ 			
				  \begin{tcolorbox}[basic] 
				      
				    \medskip
	  \centering\underline{Inconnues}
	  \medskip
	      \begin{columns}[c]
		 
		 \column{.25\textwidth}\centering
		      
		      
		         $\Pi$     $\begin{cases} 
		                    \overrightarrow{d_1} \\
		                    \supset{d_2}
		                   \end{cases}$
		 \column{.20\textwidth}\centering       
		      $\mathcal{D}(d_1,\Pi)$
		      

		  
		  \column{.25\textwidth}\centering
		      
		          $d_3$	  $\begin{cases} 
		                    \overrightarrow{d_1}\wedge \overrightarrow{d_2}  \\
		                    \ni C
		                   \end{cases}$
		  \column{.30\textwidth}\centering
				   $\begin{cases} 
				    P_1 \in d_1 \\
				    P_2 \in d_2 \\
		                    \overrightarrow{P_1P_2}\ \alpha\ \overrightarrow{d_1}\wedge \overrightarrow{d_2}  \\
		                    
		                   \end{cases}$
		      
		      
		
	      \end{columns}
				    \end{tcolorbox}
	
		\flushleft
		
		\begin{columns}[c]
	
		\column{.5\textwidth}\centering
				    
	        $d_1 \equiv \begin{cases}
		            \vect{AB} = (-2,-2,-1), \\
		            A(1,2,3).
		            \end{cases}$
					       
		\column{.5\textwidth}\centering			       		    
	        $d_2 \equiv \begin{cases}
		            \vect{CD} = (2,-1,-2), \\
		            C(0,1,7).
		            \end{cases}$
		\end{columns}\bigskip
		\centering
		$d_3 \equiv \begin{cases}
		                C, \\
		               \vect{d_3} = \vect{AB}\wedge\vect{CD}=(3,-6,6).
		               \end{cases}$ \\ \bigskip
                               \begin{columns}[c]
                                  \column{.5\textwidth}\centering
		 $$d_3 \equiv \frac{x-0}{3}=\frac{y-1}{-6}=\frac{z-7}{6}.$$
		
	
		 \column{.5\textwidth}\centering
		 \begin{align*}
		  d_3 \equiv&\ C + \lambda \vect{d_3}, \quad \forall\lambda \in \mathbb{R}\\
		      \equiv& \begin{cases}
		               x= 0 + \lambda.3\\
		               y= 1 + \lambda.(-6)\\
		               z= 7 + \lambda.6 \\
		              \end{cases}
		 \end{align*}
                 	\hfill $\qed(c)$
		\end{columns}
	\notedir{%
	.1 Équations cartésiennes et paramétriques $d_3$.
	.2 $d_3$ défini par.
	.3 Contient $C$..
	.3 Direction est $\bot$ à $d_1,d_2$..
	.4 Comment faire un produit vectoriel ?..
	.3 Avec point et vecteur on trouve.
	.4 Équations paramétriques..
	.5 $C$ plus un multiple du vecteur directeur..
	.6 On projette sur $x,y,z$..
	.4 Équations cartésiennes..
	.5 Formule d'une droite dans l'espace.
	.6 $\dfrac{(x-C_x)}{d_{3_x}}=\dfrac{(y-C_y)}{d_{3_y}}=\dfrac{(z-C_z)}{d_{3_z}}$. 
	}
    }
    
    \frame{ 			
				  \begin{tcolorbox}[basic] 
				      
				    \medskip
	  \centering\underline{Inconnues}
	  \medskip
	      \begin{columns}[c]
		 
		 \column{.25\textwidth}\centering
		      
		      
		         $\Pi$     $\begin{cases} 
		                    \overrightarrow{d_1} \\
		                    \supset{d_2}
		                   \end{cases}$
		 \column{.20\textwidth}\centering       
		      $\mathcal{D}(d_1,\Pi)$
		      

		  
		  \column{.25\textwidth}\centering
		      
		          $d_3$	  $\begin{cases} 
		                    \overrightarrow{d_1}\wedge \overrightarrow{d_2}  \\
		                    \ni C
		                   \end{cases}$
		  \column{.30\textwidth}\centering
				   $\begin{cases} 
				    P_1 \in d_1 \\
				    P_2 \in d_2 \\
		                    \overrightarrow{P_1P_2}\ \alpha\ \overrightarrow{d_1}\wedge \overrightarrow{d_2}  \\
		                    
		                   \end{cases}$
		      
		      
		
	      \end{columns}
				    \end{tcolorbox}
	
		\flushleft
		\medskip
		\begin{columns}[c]
	
		\column{.5\textwidth}\centering
		\begin{align*}			    
	        d_1 &\equiv \begin{cases}
		            \vect{AB} = (-2,-2,-1), \\
		            A(1,2,3).
		            \end{cases} \\
		 d_1 &\equiv A + \lambda\ \vect{AB}, \quad\forall\lambda\in\mathbb{R}. \\
		 P_1 &= A + p_1\ \vect{AB}, \quad p_1\in\mathbb{R}.\\[0.5em]  
		     &=\ (\ 1 - 2p_1,\ 2 -2p_1,\ 3 -p_1). 
		\end{align*}
					       
		\column{.5\textwidth}\centering			       
		 \begin{align*}
	        d_2 &\equiv \begin{cases}
		            \vect{CD} = (2,-1,-2), \\
		            C(0,1,7).
		            \end{cases} \\			    
	        d_2 &\equiv C + \lambda\ \vect{CD}, \quad\forall\lambda\in\mathbb{R}. \\ 
	        P_2 &= C + p_2\ \vect{CD}, \quad p_2\in\mathbb{R}. \\[0.5em]
		    &=\ (\ 2p_2,\ 1 -p_2,\ 7 -2p_2).
		\end{align*}
		\end{columns}		
		\vspace{5mm}
		\centering
		De (c), $\vect{d_1}\wedge\vect{d_2}=\vect{AB}\wedge\vect{CD}=(3,-6,6)$.	\\[0.5em]
		\begin{align*}
		\overrightarrow{P_1P_2}\ &\alpha\ \overrightarrow{d_1}\wedge \overrightarrow{d_2}, \\[0.5em]
		(\ 2p_2-1+2p_1,\ -1-p_2+2p_1,\ 4 -2p_2 + p_1)\ &\alpha\ (\ 3,\ -6,\ 6), \\   
		\end{align*}
    \notedir{%
    .1 Coordonnées $P_1,P_2$.
    .2 Exprimer $d_1,d_2$ en équations paramétriques.
    .3 Exprimer coordonnées de $P_1,P_2$ avec paramètres..
    .4 Exprimer vecteur $\vect{P_1P_2}$ avec paramètres..
    .2 De (c), coordonnées de $\vect{d_1}\wedge\vect{d_2}$..
    .2 Exprimer que $\vect{P_1P_2}$ multiple de $\vect{d_1}\wedge\vect{d_2}$..
    }
    }
    \frame{ 			
				  \begin{tcolorbox}[basic] 
				      
				    \medskip
	  \centering\underline{Inconnues}
	  \medskip
	      \begin{columns}[c]
		 
		 \column{.25\textwidth}\centering
		      
		      
		         $\Pi$     $\begin{cases} 
		                    \overrightarrow{d_1} \\
		                    \supset{d_2}
		                   \end{cases}$
		 \column{.20\textwidth}\centering       
		      $\mathcal{D}(d_1,\Pi)$
		      

		  
		  \column{.25\textwidth}\centering
		      
		          $d_3$	  $\begin{cases} 
		                    \overrightarrow{d_1}\wedge \overrightarrow{d_2}  \\
		                    \ni C
		                   \end{cases}$
		  \column{.30\textwidth}\centering
				   $\begin{cases} 
				    P_1 \in d_1 \\
				    P_2 \in d_2 \\
		                    \overrightarrow{P_1P_2}\ \alpha\ \overrightarrow{d_1}\wedge \overrightarrow{d_2}  \\
		                    
		                   \end{cases}$
		      
		      
		
	      \end{columns}
				    \end{tcolorbox}
	
		\flushleft
		
		\begin{align*}
		\overrightarrow{P_1P_2}\ &\alpha\ \overrightarrow{d_1}\wedge \overrightarrow{d_2}, \\[0.5em]
		(\ 2p_2-1+2p_1,\ -1-p_2+2p_1,\ 4 -2p_2 + p_1)\ &\alpha\ (\ 3,\ -6,\ 6), \\   
		\end{align*}
		$\begin{cases}
		  2p_2-1+2p_1 &=\ \alpha.3 \\
		  -1-p_2+2p_1 &=\ \alpha.(-6) \\
		  4 -2p_2 + p_1 &=\ \alpha.6
		 \end{cases}\qquad \rightarrow \qquad p_1 = 0 \text{ et } p_2 = 1$. \\ \bigskip
		 \centering $P_1 = (1,2,3) \qquad \text{ et } \qquad P_2=(2,0,5)$. \\
		 \hfill $\qed(d)$
	\notedir{%
    .1 Suite Coordonnées $P_1,P_2$.
    .2 Système de 3 équations à 3 inconnues..
    .3 Trouver coefficients $p_1,p_2$..
    .4 Trouver coordonnées de $P_1,P_2$..
    }
  }

  
	             \frame{ 
	  \enonce

	  \vfill\center
	  \underline{Résumé}\\ \flushleft
          \begin{itemize}
          \item Un plan est entièrement déterminé par un point et deux vecteurs
            directeurs,
          \item 	$\mathcal{D}(d_1,\Pi)=\mathcal{D}(A,\Pi)\ \forall A\in d_1.\quad \text{ si $d_1\parallel\Pi$}$,
          \item 	$\mathcal{D}(A,\Pi) = \dfrac{|n_{\Pi_x}A_x + n_{\Pi_y}A_y + n_{\Pi_z}A_z + d|}{|\vect{n_{\Pi}}|}$,
            \item $\vect{d_1}\wedge \vect{d_2}$ $\bot$ à $\vect{d_1}$ et  $\vect{d_2}$.
          \end{itemize}
    }
\end{document}

%%% Local Variables:
%%% mode: latex
%%% TeX-master: t
%%% End:
