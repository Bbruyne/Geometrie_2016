\documentclass[10pt]{beamer}

\usepackage[utf8]{inputenc}
\usepackage{pgfpages}
\usepackage[french]{babel}
\usepackage{dirtree}
\setbeamertemplate{note page}[plain]
\setbeameroption{show notes on second screen =left}
\AtEndNote{\vfill \begin{center} mm:hh \end{center}}
\newcommand{\notedir}[1] {
  \note{\dirtree{#1}}}
\def \ion {$^{\circ}$ }
\usepackage{tcolorbox}
\usepackage{tikz}
\usepackage{tikz-3dplot}
\usetikzlibrary{intersections,calc,,angles,quotes,through}
\usepackage{amsmath}
\usepackage{graphicx}
\usepackage{cases}
\def \heart {\textcolor{blue}{$\heartsuit$} }
\def \C {\mathcal{C}}
\def \orthog {\underline{\perp}}
\def\arcos{\operatorname{arcos}}
\def \deg {^{\circ}}

\newcommand{\vect}[1] {
  \overrightarrow{#1}}

\tcbset{%
	basic/.style={colframe=black,
		      colback=white,
		      top= 0mm,
		      bottom = 2mm,
		      boxsep=0mm
		      }
}
\tikzset{
    invisible/.style={opacity=0},
    visible on/.style={alt={#1{}{invisible}}},
    alt/.code args={<#1>#2#3}{%
      \alt<#1>{\pgfkeysalso{#2}}{\pgfkeysalso{#3}} % \pgfkeysalso doesn't change the path
    },
  }

    
\begin{document}  
    \beamertemplatenavigationsymbolsempty
    \setlength{\abovedisplayskip}{0pt}
    \setlength{\belowdisplayskip}{0pt}
    \frame{
	  
	  \frametitle{Q5 Juillet 2002.}
	  \renewcommand{\theenumi}{\alph{enumi})}
	  On se place dans l’espace muni d’un repère orthonormé $Oxyz$.
	  \begin{enumerate}
	   \item Démontrer qu’il existe une droite $d$ passant par l’origine et telle que les plans
		 d’équation \\ $$\pi \equiv (2r^2 + 6r - 2)x - (r^2 - 1)y - 2rz + r^2 + 1 = 0$$
		 où $r$ parcourt $\mathbb{R}$, soient tous parallèles à $d$.
	   \item Déterminer la perpendiculaire commune $d_1$ à $d$ et à la droite $d'$ d’équations 
		 $$d' \equiv \begin{cases} 
			x=y \\
			z=1
		        \end{cases}$$

	  \end{enumerate}

	  \vfill
	  
	  \pause
	  % hypothèses et thèse
	  \begin{tcolorbox}[basic] \smallskip
	     \centering\underline{Procédé}
	     \begin{columns}[T]
	     \column{.5\textwidth}\centering 
	     
	     \begin{enumerate}
	      \item $d \equiv \begin{cases}
	                  O(0,0,0), \\
	                  \vect{d}\in \vect{\pi}.
	                 \end{cases}$

	     \end{enumerate}
	     \column{.5\textwidth}\centering 
	     \begin{enumerate}
	      \item[b).] $d_1 \equiv \begin{cases} 
	                         \vect{d_1} = \vect{d} \wedge \vect{d'}, \\
	                         d_1 \nmid d \text{ et } d_1 \nmid d'.
	                        \end{cases}$

	     \end{enumerate}

	  \end{columns}
	  \end{tcolorbox}
	  \notedir{%
	.1 Énoncé.
	.2 Procédé : reformuler ce qu'il faut déterminer..
	.3 $d$.
	.4 Passe par origine..
	.4 Parallèle aux plans..
	.5 Son vecteur directeurs est inclus dans sous vectoriel directeur du plan..
	.3 $d_1$.
	.4 Direction est $\vect{d} \wedge \vect{d'}$ car.
	.5 À la fois $\bot\ d$ et $\bot\ d'$..
	.5 Elle intersecte $d$ et $d'$..
	}
    }

    \frame{ 
	  % résolution ex1
	
		
	     \begin{tcolorbox}[basic] \smallskip
	     \centering\underline{Procédé}
	     \begin{columns}[T]
	     \column{.5\textwidth}\centering 
	     \renewcommand{\theenumi}{\alph{enumi})} 
	     \begin{enumerate}
	      \item $d \equiv \begin{cases}
	                  O(0,0,0), \\
	                  \vect{d}\in \vect{\pi}.
	                 \end{cases}$

	     \end{enumerate}
	     \column{.5\textwidth}\centering 
	     \begin{enumerate}
	      \item[b).] $d_1 \equiv \begin{cases} 
	                         \vect{d_1} = \vect{d} \wedge \vect{d'}, \\
	                         d_1 \nmid d \text{ et } d_1 \nmid d'.
	                        \end{cases}$

	     \end{enumerate}

	  \end{columns}
	  \end{tcolorbox}
		\centering
		\underline{Résolution}\\ \flushleft
		\vspace{-5mm}
		\begin{align*}
		                    \pi \equiv &\ (2r^2 + 6r - 2)x - (r^2 - 1)y - 2rz + r^2 + 1 = 0, \\
		             \vect{\pi} \equiv &\ (2r^2 + 6r - 2)x - (r^2 - 1)y - 2rz = 0, \\
         \vect{d}=(x_0,y_0,z_0)\in \vect{\pi} : &\ (2r^2 + 6r - 2)x_0 - (r^2 - 1)y_0 - 2rz_0 = 0. \\[0.3em]
					       &\ \begin{cases}
					           2r^2x_0 - r^2y_0 &= 0, \\
					           6rx_0 - 2rz_0 &= 0, \\
					           -2x_0 + y_0 &= 0.
					          \end{cases}
		\end{align*} \\ 
		
		\medskip
		
		Une solution : $\vect{d} = (1,2,3)$. \\ \medskip
		$d \equiv \begin{cases}
	                  O(0,0,0), \\
	                  \vect{d} = (1,2,3).
	                 \end{cases}$ $\rightarrow \qquad d \equiv \dfrac{x}{1} =\ \dfrac{y}{2} =\ \dfrac{z}{3}$. \hfill $\qed(a)$
		
		%\centering\noindent\rule{2cm}{0.4pt}
	  \notedir{%
	.1 Suivre procédé.
	.2 Élément de théorie.
	.3 Droite parallèle à plan si son vecteur directeur\\ \hspace{5mm} appartient au sous-vectoriel direct.~du plan..
	.4 ie: Coordonnées du vecteur directeur vérifient l'équation du plan sans terme indépendant..
	.2 Résolution.
	.3 Équations de $d$.
	.4 Vecteur directeur de $d$ doit vérifier équation du\\ \hspace{5mm} plan sans terme indep..
	.5 Équation doit être vérifiée pour toutes valeurs de \\ \hspace{5mm}$r$..
	.6 Annuler coefficients devant $r^2$, $r$ et terme\\ \hspace{5mm} indépendant..
	.7 Peu importe la valeur de $r$, l'équation est vérifiée..
	.3 Équations de $d$ avec un point et vecteur directeur..
	.4 Son équation $\dfrac{(x-0)}{d_x}=\dfrac{(y-0)}{d_y}=\dfrac{(z-0)}{d_z}$.. 
	}
    }
    
     \frame{ 
		\onslide<+->
		 \begin{tcolorbox}[basic] \smallskip
	     \centering\underline{Procédé}
	     \begin{columns}[T]
	     \column{.5\textwidth}\centering 
	     \renewcommand{\theenumi}{\alph{enumi})} 
	     \begin{enumerate}
	      \item $d \equiv \begin{cases}
	                  O(0,0,0), \\
	                  \vect{d}\in \vect{\pi}.
	                 \end{cases}$

	     \end{enumerate}
	     \column{.5\textwidth}\centering 
	     \begin{enumerate}
	      \item[b).] $d_1 \equiv \begin{cases} 
	                         \vect{d_1} = \vect{d} \wedge \vect{d'}, \\
	                         d_1 \nmid d \text{ et } d_1 \nmid d'.
	                        \end{cases}$

	     \end{enumerate}

	  \end{columns}
	  \end{tcolorbox} \medskip
		$d \equiv \begin{cases} 
		      2x=y, \\
		      3x=z.
		      \end{cases} \hspace{-3mm} \rightarrow \begin{cases}
					       \vect{d} = (1,2,3), \\
					       D = (1,2,3).
					       \end{cases}      \hspace{-1.5mm}       
		 d' \equiv \begin{cases} 
		       x=y, \\
		       z=1.
		      \end{cases} \hspace{-3mm} \rightarrow \begin{cases}
						  \vect{d'} = (1,1,0), \\
						  D' = (0,0,1).
						  \end{cases}$\\ \medskip
						
		\begin{columns}[t]
		
		\column{.5\textwidth}\flushleft 
		\heart La perpendiculaire commune à 2 droites gauches $d,d'$ est l'intersection du plan $\alpha$ contenant $d$ et la direction $d \wedge
		d'$ avec le plan $\alpha '$ contenant $d'$ et la direction $d \wedge d'$. \\ \medskip 
		
		
		\column{.5\textwidth}\centering 
		                  \begin{figure}[h]
				  \begin{tikzpicture}[scale=0.85]
			          %projection ($(X)!(B')!(B)$)
			          %nommer chemin 'name path
			          %intersection \path [name intersections={of=d and gb,by=G}];
			          
			          %\draw[help lines] (-3,-3) grid (3,3);
			          \draw (-2,0,0) -- (2,0,0) node[above right] {$d$};
			          \draw (0,2,-2) -- (0,2,2) node[above] {$d'$};
			          \draw[visible on =<5>]  (0,0,0)  node[above right] {$d_1$} -- (0,2,0);
			          
			          \draw[visible on=<2-4>,->,xshift=5mm] (0,1,0) node[below] {$d \wedge d'$} -- (0,1.5,0); 
			          
			          \draw[dotted,visible on=<{3,5}>] (-2,0,0) -- (-2,2,0) (2,0,0) node[above left]{$\alpha$} -- (2,2,0);
			          \draw[dotted,visible on=<4-5>] (0,2,2)  node[right]{$\alpha '$} -- (0,0,2) (0,2,-2) -- (0,0,-2);
				  \end{tikzpicture}
				  \end{figure}
		\end{columns}
		\vspace{-8mm}
		\onslide<+->
		$\vect{d} \wedge \vect{d'} = (-3,3,-1)$. \\
		
		\vspace{-6mm}
		
		
		\begin{columns}[c]

		\column{.333\textwidth}\flushleft 
		\onslide<+->
		\begin{align*}
		\alpha \equiv& \begin{cases}
		                D, \\
		                \vect{d}, \\
		                \vect{d} \wedge \vect{d'}.
		               \end{cases} \\
		       \equiv&\ 11x+8y-9z=0.
		\end{align*}
		
		\column{.333\textwidth}\flushleft 
		\onslide<+->
		\begin{align*}
		\alpha' \equiv& \begin{cases}
		                D', \\
		                \vect{d'}, \\
		                \vect{d} \wedge \vect{d'}.
		               \end{cases} \\
		       \equiv&\ x-y-6z + 6=0.
		\end{align*}
		
		\column{.333\textwidth}\flushleft 
		\onslide<+->
		\begin{align*}
		 d_1      =&\ \alpha \cap \alpha ', \\
		     \equiv&\ \dfrac{x-\frac{6}{19}}{-3} = \dfrac{y-\frac{12}{19}}{3} = \dfrac{z - \frac{18}{19}}{-1}.
		\end{align*}
		\hfill $\qed(c)$\hspace{-3mm}
		\vspace{-3mm}
		
		\end{columns}
	 \notedir{%
	.1 Suivre procédé.
	.2 Élément de théorie.
	.3 Droite de direct\ion donnée, d'appui sur 2 droites est\\ \hspace{5mm} intersect\ion de 2 plans.
	.4 Le premier formé par 1\iere\ droite et la direct\ion..
	.4 Le deuxième formé par 2\ieme\ droite et la direct\ion..
	.2 Résolution.
	.3 Direction.
	.4 $d_1$ $\bot$ à $d$ et $d'$ $\rightarrow \vect{d_1} = \vect{d} \wedge \vect{d'}$..
	.3 Premier plan $\alpha$.
	.4 Formé par $d$ et direction $\vect{d} \wedge \vect{d'}$..
	.5 Formé par $D$, $\vect{d}$ et $\vect{d} \wedge \vect{d'}$..
	.4 Son équation s'obtient en annulant le déterminant..
	.3 Second plan $\alpha'$.
	.4 Formé par $d'$ et direction $\vect{d} \wedge \vect{d'}$..
	.5 Formé par $D'$, $\vect{d'}$ et $\vect{d} \wedge \vect{d'}$..
	.4 Son équation s'obtient en annulant le déterminant..
	.3 $d_1$ est intersection des 2 plans..
	.4 Équations : éliminer au moins une inconnue dans équations du système..
	}	
    }
	  
  
\end{document}
