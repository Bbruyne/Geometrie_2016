\documentclass[10pt]{beamer}

\usepackage[utf8]{inputenc}
\usepackage{pgfpages}
\usepackage[french]{babel}
\usepackage{dirtree}
\setbeamertemplate{note page}[plain]
\AtEndNote{\vfill \begin{center} mm:hh \end{center}}
\newcommand{\notedir}[1] {
  \note{\dirtree{#1}}}
\def \ion {$^{\circ}$ }
\usepackage{tcolorbox}
\usepackage{tikz}
\usepackage{tikz-3dplot}
\usetikzlibrary{intersections,calc,,angles,quotes,through}
\usepackage{amsmath}
\usepackage{graphicx}
\usepackage{cases}
\def \heart {\textcolor{blue}{$\heartsuit$} }
\def \C {\mathcal{C}}
\def \orthog {\underline{\perp}}
\def\arcos{\operatorname{arcos}}
\def \deg {^{\circ}}

\newcommand{\vect}[1] {
  \overrightarrow{#1}}

\tcbset{%
	basic/.style={colframe=black,
		      colback=white,
		      top= 0mm,
		      bottom = 2mm,
		      boxsep=0mm
		      }
}
\tikzset{
    invisible/.style={opacity=0},
    visible on/.style={alt={#1{}{invisible}}},
    alt/.code args={<#1>#2#3}{%
      \alt<#1>{\pgfkeysalso{#2}}{\pgfkeysalso{#3}} % \pgfkeysalso doesn't change the path
    },
  }

    
\begin{document}  
    \beamertemplatenavigationsymbolsempty
    \setlength{\abovedisplayskip}{0pt}
    \setlength{\belowdisplayskip}{0pt}
    \frame{
	  
	  \frametitle{Q5 Juillet 2001.}
	  \renewcommand{\theenumi}{\alph{enumi})}
	  On donne des équations cartésiennes de trois droites $d_1$, $d_2$, $d_3$ de l’espace : \\
	  $$d_1 \equiv \frac{x-1}{2}=-y=\frac{z+1}{3},\; d_2 \equiv \frac{x}{3}=y+1=\frac{z-2}{5},\; d_3 \equiv \begin{cases}
	                                                                                                     x=1 \\
	                                                                                                     y+z=2
	                                                                                                    \end{cases}$$
	   \begin{enumerate}
	    \item Donner une équation du plan $\alpha$ contenant $d_1$ et parallèle à $d_3$.
	    \item Donner une équation du plan $\beta$ contenant $d_2$ et parallèle à $d_3$
	    \item Donner les équations de la droite $d$ s’appuyant sur $d_1$ et $d_2$ et parallèle à $d_3$.
	   \end{enumerate}
                                                                                                 

	  \vfill
	  
	  \pause
	  
	   \begin{tcolorbox}[basic] \smallskip
	     \centering\underline{Procédé}
	     \begin{columns}[c]
	     \column{.333\textwidth}\centering 
	     
	     \renewcommand{\theenumi}{\alph{enumi})}
	     \begin{enumerate}
	      \item $\alpha \equiv \begin{cases}
				    \supset d_1, \\
				    \parallel d_3.
				   \end{cases}$	      
	     \end{enumerate}
	     
	     \column{.333\textwidth}\centering 
	     
	     \begin{enumerate}
	      \item[b).] $\beta \equiv \begin{cases}
				    \supset d_2, \\
				    \parallel d_3.
				   \end{cases}$   
	     \end{enumerate}
	     \column{.333\textwidth}\centering 
	     
	     \begin{enumerate}
	      \item[c).]$d  \equiv \begin{cases}
				    d \nmid d_1, \\
				    d \nmid d_2, \\
				    \parallel d_3.
				   \end{cases}$

	     \end{enumerate}

	  \end{columns}
	  \end{tcolorbox}
	   \notedir{%
	.1 Énoncé.
	.2 Procédé : reformuler ce qu'il faut déterminer..
	}
    }

    \frame{ 			
				  \begin{tcolorbox}[basic] 
				      
				    \smallskip
				    \centering\underline{Procédé}
				    \begin{columns}[c]
				      \column{.333\textwidth}\centering 
				      
				      \renewcommand{\theenumi}{\alph{enumi})}
				      \begin{enumerate}
					\item $\alpha \equiv \begin{cases}
							      \supset d_1, \\
							      \parallel d_3.
							    \end{cases}$	      
				      \end{enumerate}
				      
				      \column{.333\textwidth}\centering 
				      
				      \begin{enumerate}
					\item[b).] $\beta \equiv \begin{cases}
							      \supset d_2, \\
							      \parallel d_3.
							    \end{cases}$   
				      \end{enumerate}
				      \column{.333\textwidth}\centering 
				      
				      \begin{enumerate}
					\item[c).]$d  \equiv \begin{cases}
							      d \nmid d_1, \\
							      d \nmid d_2, \\
							      \parallel d_3.
							    \end{cases}$

				      \end{enumerate}

				    \end{columns}
				    \end{tcolorbox}
	
		\centering\underline{Résolution}\\ \flushleft
		\medskip
		\begin{columns}[c]
	
		\column{.333\textwidth}\centering
		\begin{align*}
	        d_1 &\equiv \begin{cases} 
		      \frac{x-1}{2} = -y, \\
		      -y = \frac{z+1}{3}.
			    \end{cases} \\ 			    
	        d_1 &\equiv \begin{cases}
		            \vect{d_1} = (2,-1,3), \\
		            D_1 = (1,0,-1).
		            \end{cases}      
		\end{align*}
					       
		\column{.333\textwidth}\centering			       
		 \begin{align*}
	        d_2 &\equiv \begin{cases} 
		      \frac{x}{3}=y+1, \\
		      y+1=\frac{z-2}{5}.
			    \end{cases} \\ 			    
	        d_2 &\equiv \begin{cases}
		            \vect{d_2} = (3,1,5), \\
		            D_2 = (0,-1,2).
		            \end{cases}      
		\end{align*}
	
		\column{.333\textwidth}\centering
		\begin{align*}
	        d_3 &\equiv \begin{cases} 
		      x=1, \\
		      y+z=2.
			    \end{cases} \\ 			    
	        d_3 &\equiv \begin{cases}
		            \vect{d_3} = (0,1,-1), \\
		            D_3 = (1,1,1).
		            \end{cases}      
		\end{align*}
		\end{columns}
		\bigskip
		\begin{columns}[c]

		\column{.5\textwidth} 	
		\begin{align*}
		\alpha \equiv& \begin{cases}
		                D_1, \\
		                \vect{d_1}, \\
		                \vect{d_3}.
		               \end{cases} \\
		       \equiv&\ x-y-z-2=0.
		\end{align*} \hfill $\qed(a)$
		
		\column{.5\textwidth}
		\begin{align*}
		\beta \equiv& \begin{cases}
		                D_2, \\
		                \vect{d_2}, \\
		                \vect{d_3}.
		               \end{cases} \\
		       \equiv&\ 2x-y-z+1=0.
		\end{align*} \hfill $\qed(b)$
		
		\end{columns}	\hfill
		\notedir{%
	.1 Suivre procédé.
	.2 Élément de théorie.
	.3 Exprimer $d_1,d_2,d_3$ avec un vecteur direct.~et un point.. 
	.4 Coordonnées d'un vecteur directeur sont solutions du système sans les termes indépendants..
	.4 Coordonnées d'un point sont solutions du système avec les termes indépendants..
	.2 Résolution.
	.3 Équation de $\alpha$.
	.4 $\alpha$ est déterminé par un point et 2 vecteurs direct..
	.5 Son équation s'obtient en annulant le déterminant:.
	.6 1\iere\ colonne : $x - D_{1_x}$,$y - D_{1_y}$, $z - D_{1_z}$..
	.6 2\ieme\ colonne : coordonnée de $\vect{d_1}$..
	.6 3\ieme\ colonne : coordonnée de $\vect{d_3}$..	
	.3 Équation de $\beta$.
	.4 $\beta$ est déterminé par un point et 2 vecteurs direct..
	.5 Son équation s'obtient en annulant le déterminant:.
	.6 1\iere\ colonne : $x - D_{2_x}$,$y - D_{2_y}$, $z - D_{2_z}$..
	.6 2\ieme\ colonne : coordonnée de $\vect{d_2}$..
	.6 3\ieme\ colonne : coordonnée de $\vect{d_3}$..	
	}
    }
	\frame{ 			
				  \begin{tcolorbox}[basic] 
				      
				    \smallskip
				    \centering\underline{Procédé}
				    \begin{columns}[c]
				      \column{.333\textwidth}\centering 
				      
				      \renewcommand{\theenumi}{\alph{enumi})}
				      \begin{enumerate}
					\item $\alpha \equiv \begin{cases}
							      \supset d_1, \\
							      \parallel d_3.
							    \end{cases}$	      
				      \end{enumerate}
				      
				      \column{.333\textwidth}\centering 
				      
				      \begin{enumerate}
					\item[b).] $\beta \equiv \begin{cases}
							      \supset d_2, \\
							      \parallel d_3.
							    \end{cases}$   
				      \end{enumerate}
				      \column{.333\textwidth}\centering 
				      
				      \begin{enumerate}
					\item[c).]$d  \equiv \begin{cases}
							      d \nmid d_1, \\
							      d \nmid d_2, \\
							      \parallel d_3.
							    \end{cases}$

				      \end{enumerate}

				    \end{columns}
				    \end{tcolorbox}
	
		\centering
		\medskip
		\begin{columns}[c]
		 \column{.5\textwidth}\centering 
		 $\begin{cases}
		   d \nmid d_1,\\
		   d \parallel d_3.
		  \end{cases} 
		  \rightarrow 
		  \begin{cases}
		   d \nmid \alpha,\\
		   d \parallel \alpha.
		  \end{cases}
		  \rightarrow
		  d \subset \alpha$.

		 \column{.5\textwidth}\centering 
		 $\begin{cases}
		   d \nmid d_2, \\
		   d \parallel d_3.
		  \end{cases} 
		  \rightarrow
		  \begin{cases}
		   d \nmid \beta,\\
		   d \parallel \beta.
		  \end{cases}
		  \rightarrow 
		  d \subset \beta$.	 
		\end{columns} \bigskip
		\begin{align*}
		d=\alpha \cap \beta =& \begin{cases}
		                        x-y-z-2=0, \\
		                        2x-y-z+1=0.
		                       \end{cases} \\
			            =& \begin{cases}
		                        x=-3, \\
		                        y+z=-5.
		                       \end{cases}
		\end{align*} \hfill $\qed(c)$
	\notedir{%
	.1 Suivre procédé.
	.2 Élément de théorie.
	.3 Droite de direct\ion donnée, d'appui sur 2 droites est\\ \hspace{5mm} intersect\ion de 2 plans.
	.4 Le premier formé par 1\iere\ droite et la direct\ion..
	.4 Le deuxième formé par 2\ieme\ droite et la direct\ion..
	.2 Résolution.
	.3 Preuve élément théorie.
	.4 $d$ intersecte $d_1$ $\rightarrow$ $d$ intersecte plan $\alpha$..
	.4 $d$ parallèle à $d_3$ $\rightarrow$ $d$ parallèle à plan $\alpha$..
	.5 Si $d$ intersecte et est parallèle à $\alpha$, elle est contenue dans $\alpha$..
	.4 $d$ intersecte $d_2$ $\rightarrow$ $d$ intersecte plan $\beta$..
	.4 $d$ parallèle à $d_3$ $\rightarrow$ $d$ parallèle à plan $\beta$..
	.5 Si $d$ intersecte et est parallèle à $\beta$, elle est contenue dans $\beta$..
	.4 $d$ est intersection de $\alpha$ et $\beta$..
	.5 Équations : éliminer au moins une inconnue dans équations du système..
	}
    }  
  
\end{document}
