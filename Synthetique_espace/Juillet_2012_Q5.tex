\documentclass[10pt]{beamer}

\usepackage[utf8]{inputenc}
\usepackage{tcolorbox}
\usepackage{tikz}
\usetikzlibrary{intersections}
\usepackage{amsmath}


\tcbset{%
	basic/.style={colframe=black,
		      colback=white,
		      top= 0mm,
		      bottom = 2mm,
		      boxsep=0mm
		      }
}

    
\begin{document}  
    \beamertemplatenavigationsymbolsempty
    
    \frame{
	  
	  \frametitle{Q1 Juillet 2012.}
	  On considère un tétraèdre $ABCD$ et un plan parallèle à sa base $ABC$,
	  qui coupe les arêtes $[AD]$, $[BD]$ et $[CD]$ en des points notés respectivement $A'$ , $B'$ et $C'$.
	  Dans le triangle $ABC$, les milieux des côtés $[BC]$,
	  $[AC]$ et $[AB]$ sont respectivement notés $P$, $Q$ et $R$. \\
	  Démontrer que les droites $A'P$, $B'Q$ et $C'R$ sont concourantes.
	  \vfill
	  
	  \pause
	  % hypothèses et thèse
	  \begin{tcolorbox}[basic] 
	      \begin{columns}[t]
		 
		 \column{.5\textwidth}\centering
		      
		      \underline{Hypothèses} 
		      \begin{itemize}
%TODO symbole parallèle
		      \item $ABC//A'B'C'$,
		      \item $|BP|=|PC|$ \\
			    $|AQ|=|QC|$ \\
			    $|AR|=|RB|$.
		      \end{itemize}

		  
		  \column{.5\textwidth}\centering
		      
		      \underline{Thèse} \\		
		      \smallskip
		      $A'P$, $B'Q$ et $C'R$ sont concourantes.
		
	      \end{columns}
	  \end{tcolorbox}
    }

    \frame{ 
	  % résolution ex1
	  \begin{columns}[t]
		\column{.5\textwidth}\centering 
		

			\underline{Dessin}\\
			
				  \begin{figure}[h]
				  \begin{tikzpicture}[scale=0.8]
%TODO dessin 3D					
				  \end{tikzpicture}
				  \end{figure}
			
				  \begin{tcolorbox}[basic] 
				      
				    \smallskip
				    \underline{Hypothèses} 
				    \begin{enumerate}
%TODO symbole parallèle
				    \item $ABC // A'B'C'$,
				    \item $|BP|=|PC|$ \\
					  $|AQ|=|QC|$ \\
					  $|AR|=|RB|$. 
				    \end{enumerate}
							      
				    \underline{Thèse} \\
				    \smallskip
				    $A'P$, $B'Q$ et $C'R$ sont concourantes.
				    \end{tcolorbox}
		
		
		\column{.5\textwidth}\flushleft
		
		\underline{Résolution}\\
%TODO coeur en bleu		
		\textit{Cas particulier (1): $ABC$ = $A'B'C'$} \\ \medskip		
		$\heartsuit$ Les médianes d'un triangle sont concourantes.
		\bigskip
		
		\textit{Cas particulier (2): $ABC$ contient $D$} \\ \medskip
		Les droites sont concourantes en D.
		\bigskip
		
		\textit{Cas général :} \\ medskip
		\begin{enumerate}
		 \item $AB//A'B'$,
		 \item $AB//PQ$ par Thalès.
		\end{enumerate}	
		\centering $A'B'// PQ$
		
		\begin{itemize} 
		\item[$\heartsuit$]
		\end{itemize}
		
		
		\medskip 
	        \flushright $\qed$

   
	   \end{columns}
    
    
    
    }
	  
  
\end{document}
