\documentclass[10pt]{beamer}

\usepackage[utf8]{inputenc}
\usepackage{tcolorbox}
\usepackage{tikz}
\usepackage{tikz-3dplot}
\usetikzlibrary{intersections,calc}
\usepackage{amsmath}
\usepackage{graphicx}
\usepackage{cases}

\def \heart {\textcolor{blue}{$\heartsuit$} }
\def \C {$\mathcal{C}$}


\tcbset{%
	basic/.style={colframe=black,
		      colback=white,
		      top= 0mm,
		      bottom = 2mm,
		      boxsep=0mm
		      }
}

    
\begin{document}  
    \beamertemplatenavigationsymbolsempty
    \setlength{\abovedisplayskip}{0pt}
    \setlength{\belowdisplayskip}{0pt}
    
    
    \frame{
	  
	  \frametitle{Q4 Juillet 2001.}
	  Dans un repère orthonormé de l’espace, on considère la droite $d_1$ , pas-
	  sant par les points $A$ et $B$ respectivement de coordonnées $(1, 2, 3)$ et
	  $(-1, 0, 2)$ et la droite $d_2$ , passant par les points $C$,$D$ respectivement
	  de coordonnées $(0, 1, 7)$ et $(2, 0, 5)$.
	  \renewcommand{\theenumi}{(\alph{enumi})}
	  \begin{enumerate}
	   \item Déterminer l’équation cartésienne du plan $\Pi$ parallèle à la droite
		 $d_1$ et contenant la droite $d_2$.
	   \item Déterminer la distance entre la droite $d_1$ et le plan $\Pi$.
	   \item Déterminer des équations paramétriques et des équations carté-
		 siennes de la droite $d_3$ passant par $C$ et orthogonale à $d_1$ et $d_2$.
	   \item Déterminer un point $P_1$ de $d_1$ et point $P_2$ de $d_2$ tels que le vecteur
		 joignant $P_1$ à $P_2$ soit orthogonal à $d_1$ et à $d_2$ .
	  \end{enumerate}

	  
	  \vfill
	  
	  \pause
	  % hypothèses et thèse
	  \begin{tcolorbox}[basic] 
	  \medskip
	  \centering\underline{Inconnues}
	  \medskip
	      \begin{columns}[c]
		 
		 \column{.25\textwidth}\centering
		      
		      
		         $\Pi$     $\begin{cases} 
		                    \overrightarrow{d_1} \\
		                    \supset{d_2}
		                   \end{cases}$
		 \column{.20\textwidth}\centering       
		      $\mathcal{D}(d_1,\Pi)$
		      

		  
		  \column{.25\textwidth}\centering
		      
		          $d_3$	  $\begin{cases} 
		                    \overrightarrow{d_1}\wedge \overrightarrow{d_2}  \\
		                    \ni C
		                   \end{cases}$
		  \column{.30\textwidth}\centering
				   $\begin{cases} 
				    P_1 \in d_1 \\
				    P_2 \in d_2 \\
		                    \overrightarrow{P_1P_2}=\overrightarrow{d_1}\wedge \overrightarrow{d_2}  \\
		                    
		                   \end{cases}$
		      
		      
		
	      \end{columns}
	  \end{tcolorbox}
    }

    \frame{ 
	  % résolution ex1
	  \begin{columns}[t]
		\column{.5\textwidth}\centering 
		

			\underline{Dessin}\\
			
				  \begin{figure}[h]
				  \begin{tikzpicture}[scale=0.8]
					
				  \end{tikzpicture}
				  \end{figure}
			
				  \begin{tcolorbox}[basic] 
				      
				    \smallskip
				    \underline{Hypothèses} 
				    \begin{enumerate}
				    \item  
				    \end{enumerate}
							      
				    \underline{Thèse} \\
				    \smallskip
				    $ $.
				    \end{tcolorbox}
		
		
		\column{.5\textwidth}\flushleft
		
		\underline{Résolution}\\
		
		\begin{enumerate}
		 \item 
		\end{enumerate}	
		
		\begin{itemize} 
		\item[$\heartsuit$]
		\end{itemize}
		
		
		\medskip 
	        \flushright $\qed$

   
	   \end{columns}
    
    
    
    }
	  
  
\end{document}
