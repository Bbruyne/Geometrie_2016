\documentclass[10pt]{beamer}

\usepackage[utf8]{inputenc}
\usepackage{tcolorbox}
\usepackage{tikz}
\usepackage{tikz-3dplot}
\usetikzlibrary{intersections,calc,,angles,quotes,through}
\usepackage{amsmath}
\usepackage{graphicx}
\usepackage{cases}
\def \heart {\textcolor{blue}{$\heartsuit$} }
\def \C {\mathcal{C}}
\def \orthog {\underline{\perp}}
\def\arcos{\operatorname{arcos}}
\def \deg {^{\circ}}

\newcommand{\vect}[1] {
  \overrightarrow{#1}}

\tcbset{%
	basic/.style={colframe=black,
		      colback=white,
		      top= 0mm,
		      bottom = 2mm,
		      boxsep=0mm
		      }
}
\tikzset{
    invisible/.style={opacity=0},
    visible on/.style={alt={#1{}{invisible}}},
    alt/.code args={<#1>#2#3}{%
      \alt<#1>{\pgfkeysalso{#2}}{\pgfkeysalso{#3}} % \pgfkeysalso doesn't change the path
    },
  }

    
\begin{document}  
    \beamertemplatenavigationsymbolsempty
    \setlength{\abovedisplayskip}{0pt}
    \setlength{\belowdisplayskip}{0pt}
    \frame{
	  
	  \frametitle{Exercice supplémentaire.}
	  \renewcommand{\theenumi}{\alph{enumi})}
	  Dans un plan $R$, on considère quatre points distincts $A$, $A'$, $B$, $B'$ tels que :
	  \begin{itemize}
	   \item les droites $AB$ et $A'B'$ soient sécantes en un point noté $O$,
	   \item les droites $AB'$ et$A'B$ soient sécantes en un point noté $O'$,
	   \item les droites $AA'$ et $OO'$ soient sécantes en un point noté $F$.
	  \end{itemize}
	  Soient $\mathcal{D}$ et $\mathcal{D}'$ deux droites parallèles coupant le plan $R$ en
	  respectivement $B$ et $B'$. À tout point $M$ de $\mathcal{D}$, on associe le point $M'$,
	  intersection de $\mathcal{D}'$ et du plan $AMA'$. 
	  
	  \bigskip
	  \begin{enumerate}
	   \item Justifier que $AM$ et $A'M'$ sont sécantes (en un point noté $I$).
		 Déterminer le lieu de $I$ lorsque $M$ parcourt $\mathcal{D}$.
	   \item Justifier que $A'M$ et $AM'$ sont sécantes (en un point noté $J$).
		 Déterminer le lieu de $J$ lorsque $M$ parcourt $\mathcal{D}$.
	   \item Démontrer que la droite $IJ$ passe par un point fixe $F$ lorsque le point
		 $M$ parcourt $\mathcal{D}$.
	  \end{enumerate}

	   
	
	  \vfill
	  
	  \pause
	  % hypothèses et thèse
	  \begin{tcolorbox}[basic] 
	      \begin{columns}[t]
		 
		 \column{.5\textwidth}\centering
		      
		      \underline{Hypothèses} 
		      \begin{itemize}
		      \item 
		      \end{itemize}

		  
		  \column{.5\textwidth}\centering
		      
		      \underline{Thèse} \\
		      \smallskip
		      $ $.
		
	      \end{columns}
	  \end{tcolorbox}
    }

    \frame{ 
	  % résolution ex1
	  \begin{columns}[t]
		\column{.5\textwidth}\centering 
		

			\underline{Dessin}\\
			
				  \begin{figure}[h]
				  \begin{tikzpicture}[scale=0.8]
			          %projection ($(X)!(B')!(B)$)
			          %nommer chemin 'name path
			          %intersections \path [name intersections={of=d and gb,name=G}];
			          %intersection \path [name intersections={of=d and gb,by=G}];
			          %animation  \draw[visible on=<1>] 
				  %           \draw[visible on=<{2,4}>]
				  %angle arc[radius = 6mm, start angle= 180, end angle= 225] node [below left,pos=0.3]{$\alpha$}
				  %angle \pic [draw,"$\alpha$", angle eccentricity=1.5] {angle = A'--A--B};
				  %perpendiculaire ($(A')!3cm!-90:(A)$)
				  %cercle par point \node [draw] at (A) [circle through=(B)] {};
				  \end{tikzpicture}
				  \end{figure}
			
				  \begin{tcolorbox}[basic] 
				      
				    \smallskip
				    \underline{Hypothèses} 
				    \begin{enumerate}
				    \item  
				    \end{enumerate}
							      
				    \underline{Thèse} \\
				    \smallskip
				    $ $.
				    \end{tcolorbox}
		
		
		\column{.5\textwidth}\centering
		
		\underline{Résolution}\\ \flushleft
		
		\begin{enumerate}
		 \item 
		\end{enumerate}	
		
		\heart
		
		%\centering\noindent\rule{2cm}{0.4pt}
	   \end{columns} 
    }
	  
  
\end{document}
